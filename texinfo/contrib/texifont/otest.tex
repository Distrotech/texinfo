\catcode`\@=11
\input ienc
\input oenc
\input fsel


%\tracingfonts2
\def\msg#1{\medskip\leftline{#1}\message{^^J^^J** #1}\ignorespaces}



\msg{Copyright and registered}

x\copyright\ x\registeredsymbol

{\setfontencoding{T1}\setfontfamily{LMRoman}\selectfont
x\copyright\ x\registeredsymbol}



% Declare a test font family, with very different fonts (visually) for
% OT1 and T1.
\declarefontfamily TestFam 1000 1200
\fontmapfamily TestFam OT1 CMSans
\fontmapfamily TestFam T1 URWPalladio
\fontmapfamily TestFam TS1 LMRoman
\fontmapfamily TestFam OML CMRoman
\fontmapfamily TestFam OMS CMRoman
\fontmapfamily TestFam T2A LHRoman

% Define some encoding-specific glyphs.
\fenc@begin{OT1}
\fAcc\a{19}% \'
\fCmp\a e{\'e}
\fCmp\a E{\f@restore@enc\'e}
\fCmp\a g{\'g}
\fCmp\a G{\f@restore@enc\'g}
\fCmp\a m{\b{oo}}
\fCmp\a M{\f@restore@enc\b{oo}}
\fCmp\a n{\b \copyright}
\fCmp\a N{\f@restore@enc\b \copyright}
\fCmd\copyright{WrongEncoding}
%
\fenc@begin{OML}
\fAcc\b{127}% \tieafter

\catcode`\@=12



\msg{Dollar, pounds, euro (in OT1)}

\textdollar{\itshape\textdollar}
{\bfseries\textdollar{\itshape\textdollar}}

% There's no bold `ui' font, so we don't get a bold \pounds.
\pounds{\itshape\pounds}
{\bfseries\pounds{\itshape\pounds}}

\euro{\itshape\euro}
{\bfseries\euro{\itshape\euro}}



\msg{Cache}
% To see this test working, you must uncomment the messages in
% \f@search@command and \f@search@composite (but this will break
% accents, so for other test comment them out again).

% The cache is global and does not depend on font shape.
{\bf
  \textregistered\textless\'a\texttimes\'g \tieaccent{oo}
  \message{^^J*}
}\quad
\textregistered\textless\'a\texttimes\'g \tieaccent{oo}
\message{^^J**}

\documentencoding ISO-8859-1
% Now the new cache will be filled because \documentencoding changed
% the current font encoding list.
{\bf
  \textregistered\textless\'a\texttimes\'g \tieaccent{oo}
  \message{^^J*}
}\quad
\textregistered\textless\'a\texttimes\'g \tieaccent{oo}
\message{^^J**}

\setfontencoding{T1}\setfontfamily{TestFam}\selectfont
% Again, another cache because encodings list of TestFam is different
% from that of CMRoman.
{\bf
  \textregistered\textless\'a\texttimes\'g \tieaccent{oo} \CYRZH
  \message{^^J*}
}\quad
\textregistered\textless\'a\texttimes\'g \tieaccent{oo} \CYRZH
\message{^^J**}


% We are now in ISO-8859-1 input encoding and T1 encoding of TestFam.


\msg{Accents}
\'{}
\'a
\'�
\'\copyright
\quad
% |a| in T1 (the current encoding), the accent in OT1 (the only
% encoding containing |\a|).
\a a
\a �
\a \copyright
\quad
% Argument of the accent can contain an expandable command (including
% an active character) only as a first token.  But even then, it's
% best if it's the only token in the argument, because |\bf a| and
% |\bfa| will not be distinguished by the cache.
\a{\bf c}
%\a{\bf �}% Doesn't work.
\def\boldohat{\bf �}%
\a\boldohat
%\a{\bf \ae}% Doesn't work.
\def\boldae{\bf\ae}%
\a\boldae



\msg{Composites}
\'e
\'g
\quad
% Improperly defined composite -- |\a e| does not restore the original
% encoding before calling another glyph macro (|\'e|), which results
% in |e| being typeset in the encoding of the |\a e| composite (OT1).
\a e
\a g
% Now the original encoding is properly restored -- therefore |\'e|
% is called from T1; since T1 contains the composite |\'e|, we get
% that.  For |\'g| we get an accent since there's no such composite.
\a E
\a G
\quad
% A more complex example with the actual accent coming from yet
% another encoding.
\a m
\a M
\quad
% And now even the accent's argument is from a different encoding.
\a n
\a N

\bye

% Local variables:
% compile-command: "tex --interact=nonstopmode otest.tex"
% coding: latin-1
% End:
