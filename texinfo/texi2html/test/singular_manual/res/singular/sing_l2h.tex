% This document was automatically generated by the l2h extenstion of texi2html
% DO NOT EDIT !!!
\documentclass{article}
\usepackage{html}
\begin{document}
\begin{rawhtml}

<!-- l2h_begin sing_l2h 1 -->
\end{rawhtml}
H.\ Sch\"onemann; Humboldt-Universit\"at
\begin{rawhtml}
<!-- l2h_end sing_l2h 1 -->

\end{rawhtml}
\begin{rawhtml}

<!-- l2h_begin sing_l2h 2 -->
\end{rawhtml}
$3 \times 3$
\begin{rawhtml}
<!-- l2h_end sing_l2h 2 -->

\end{rawhtml}
\begin{rawhtml}

<!-- l2h_begin sing_l2h 3 -->
\end{rawhtml}
$0$
\begin{rawhtml}
<!-- l2h_end sing_l2h 3 -->

\end{rawhtml}
\begin{rawhtml}

<!-- l2h_begin sing_l2h 4 -->
\end{rawhtml}
$Q$
\begin{rawhtml}
<!-- l2h_end sing_l2h 4 -->

\end{rawhtml}
\begin{rawhtml}

<!-- l2h_begin sing_l2h 5 -->
\end{rawhtml}
$a$
\begin{rawhtml}
<!-- l2h_end sing_l2h 5 -->

\end{rawhtml}
\begin{rawhtml}

<!-- l2h_begin sing_l2h 6 -->
\end{rawhtml}
$Z/32003[x,y,z]$
\begin{rawhtml}
<!-- l2h_end sing_l2h 6 -->

\end{rawhtml}
\begin{rawhtml}

<!-- l2h_begin sing_l2h 7 -->
\end{rawhtml}
$x^3$
\begin{rawhtml}
<!-- l2h_end sing_l2h 7 -->

\end{rawhtml}
\begin{rawhtml}

<!-- l2h_begin sing_l2h 8 -->
\end{rawhtml}
$K[x,y,z]/\hbox{\rm jacob}(f)$.
\begin{rawhtml}
<!-- l2h_end sing_l2h 8 -->

\end{rawhtml}
\begin{rawhtml}

<!-- l2h_begin sing_l2h 9 -->
\end{rawhtml}
$12-4=8$.
\begin{rawhtml}
<!-- l2h_end sing_l2h 9 -->

\end{rawhtml}
\begin{rawhtml}

<!-- l2h_begin sing_l2h 10 -->
\end{rawhtml}
$1 \times 1$-minors
\begin{rawhtml}
<!-- l2h_end sing_l2h 10 -->

\end{rawhtml}
\begin{rawhtml}

<!-- l2h_begin sing_l2h 11 -->
\end{rawhtml}
$1 \times 1$-minors,
\begin{rawhtml}
<!-- l2h_end sing_l2h 11 -->

\end{rawhtml}
\begin{rawhtml}

<!-- l2h_begin sing_l2h 12 -->
\end{rawhtml}
$MD$
\begin{rawhtml}
<!-- l2h_end sing_l2h 12 -->

\end{rawhtml}
\begin{rawhtml}

<!-- l2h_begin sing_l2h 13 -->
\end{rawhtml}
$r^3/MD$.
\begin{rawhtml}
<!-- l2h_end sing_l2h 13 -->

\end{rawhtml}
\begin{rawhtml}

<!-- l2h_begin sing_l2h 14 -->
\end{rawhtml}
$M = r^3/U$,
\begin{rawhtml}
<!-- l2h_end sing_l2h 14 -->

\end{rawhtml}
\begin{rawhtml}

<!-- l2h_begin sing_l2h 15 -->
\end{rawhtml}
$M$.
\begin{rawhtml}
<!-- l2h_end sing_l2h 15 -->

\end{rawhtml}
\begin{rawhtml}

<!-- l2h_begin sing_l2h 16 -->
\end{rawhtml}
$\hbox{ann}(M) = \{a \mid aM = 0 \}$
\begin{rawhtml}
<!-- l2h_end sing_l2h 16 -->

\end{rawhtml}
\begin{rawhtml}

<!-- l2h_begin sing_l2h 17 -->
\end{rawhtml}
$\{ a \mid ar^3 \in U \}$.
\begin{rawhtml}
<!-- l2h_end sing_l2h 17 -->

\end{rawhtml}
\begin{rawhtml}

<!-- l2h_begin sing_l2h 18 -->
\end{rawhtml}
$U \colon r^3 $.
\begin{rawhtml}
<!-- l2h_end sing_l2h 18 -->

\end{rawhtml}
\begin{rawhtml}

<!-- l2h_begin sing_l2h 19 -->
\end{rawhtml}
$n$
\begin{rawhtml}
<!-- l2h_end sing_l2h 19 -->

\end{rawhtml}
\begin{rawhtml}

<!-- l2h_begin sing_l2h 20 -->
\end{rawhtml}
$n=0$
\begin{rawhtml}
<!-- l2h_end sing_l2h 20 -->

\end{rawhtml}
\begin{rawhtml}

<!-- l2h_begin sing_l2h 21 -->
\end{rawhtml}
$R^5$
\begin{rawhtml}
<!-- l2h_end sing_l2h 21 -->

\end{rawhtml}
\begin{rawhtml}

<!-- l2h_begin sing_l2h 22 -->
\end{rawhtml}
$(z^3,0,-y+4z,x+2z,0)$ and $(-xyz-y^2z-4xz^2+16z^3,-y^2,48z,48z,x+y-z)$.
\begin{rawhtml}
<!-- l2h_end sing_l2h 22 -->

\end{rawhtml}
\begin{rawhtml}

<!-- l2h_begin sing_l2h 23 -->
\end{rawhtml}
finite fields $Z/p$, $p$ a prime $\le 2147483629$,
\begin{rawhtml}
<!-- l2h_end sing_l2h 23 -->

\end{rawhtml}
\begin{rawhtml}

<!-- l2h_begin sing_l2h 24 -->
\end{rawhtml}
finite fields $\hbox{GF}(p^n)$ with $p^n$ elements, $p$ a prime, $p^n \le 2^{15}$,
\begin{rawhtml}
<!-- l2h_end sing_l2h 24 -->

\end{rawhtml}
\begin{rawhtml}

<!-- l2h_begin sing_l2h 25 -->
\end{rawhtml}
$Z/p$
\begin{rawhtml}
<!-- l2h_end sing_l2h 25 -->

\end{rawhtml}
\begin{rawhtml}

<!-- l2h_begin sing_l2h 26 -->
\end{rawhtml}
$Q[a,b,c,d]$
\begin{rawhtml}
<!-- l2h_end sing_l2h 26 -->

\end{rawhtml}
\begin{rawhtml}

<!-- l2h_begin sing_l2h 27 -->
\end{rawhtml}
$Z/7[x,y,z]$
\begin{rawhtml}
<!-- l2h_end sing_l2h 27 -->

\end{rawhtml}
\begin{rawhtml}

<!-- l2h_begin sing_l2h 28 -->
\end{rawhtml}
$Z/7[x_1,\ldots,x_6]$
\begin{rawhtml}
<!-- l2h_end sing_l2h 28 -->

\end{rawhtml}
\begin{rawhtml}

<!-- l2h_begin sing_l2h 29 -->
\end{rawhtml}
$x_1,x_2,x_3$
\begin{rawhtml}
<!-- l2h_end sing_l2h 29 -->

\end{rawhtml}
\begin{rawhtml}

<!-- l2h_begin sing_l2h 30 -->
\end{rawhtml}
$x_4,x_5,x_6$:
\begin{rawhtml}
<!-- l2h_end sing_l2h 30 -->

\end{rawhtml}
\begin{rawhtml}

<!-- l2h_begin sing_l2h 31 -->
\end{rawhtml}
$(Q[a,b,c])[x,y,z]$
\begin{rawhtml}
<!-- l2h_end sing_l2h 31 -->

\end{rawhtml}
\begin{rawhtml}

<!-- l2h_begin sing_l2h 32 -->
\end{rawhtml}
$(x,y,z)$
\begin{rawhtml}
<!-- l2h_end sing_l2h 32 -->

\end{rawhtml}
\begin{rawhtml}

<!-- l2h_begin sing_l2h 33 -->
\end{rawhtml}
$Q[x,y,z]$
\begin{rawhtml}
<!-- l2h_end sing_l2h 33 -->

\end{rawhtml}
\begin{rawhtml}

<!-- l2h_begin sing_l2h 34 -->
\end{rawhtml}
$x$
\begin{rawhtml}
<!-- l2h_end sing_l2h 34 -->

\end{rawhtml}
\begin{rawhtml}

<!-- l2h_begin sing_l2h 35 -->
\end{rawhtml}
$y$
\begin{rawhtml}
<!-- l2h_end sing_l2h 35 -->

\end{rawhtml}
\begin{rawhtml}

<!-- l2h_begin sing_l2h 36 -->
\end{rawhtml}
$z$
\begin{rawhtml}
<!-- l2h_end sing_l2h 36 -->

\end{rawhtml}
\begin{rawhtml}

<!-- l2h_begin sing_l2h 37 -->
\end{rawhtml}
$K[x,y,z]$
\begin{rawhtml}
<!-- l2h_end sing_l2h 37 -->

\end{rawhtml}
\begin{rawhtml}

<!-- l2h_begin sing_l2h 38 -->
\end{rawhtml}
$K=Z/7(a,b,c)$
\begin{rawhtml}
<!-- l2h_end sing_l2h 38 -->

\end{rawhtml}
\begin{rawhtml}

<!-- l2h_begin sing_l2h 39 -->
\end{rawhtml}
$Z/7$
\begin{rawhtml}
<!-- l2h_end sing_l2h 39 -->

\end{rawhtml}
\begin{rawhtml}

<!-- l2h_begin sing_l2h 40 -->
\end{rawhtml}
$b$
\begin{rawhtml}
<!-- l2h_end sing_l2h 40 -->

\end{rawhtml}
\begin{rawhtml}

<!-- l2h_begin sing_l2h 41 -->
\end{rawhtml}
$c$
\begin{rawhtml}
<!-- l2h_end sing_l2h 41 -->

\end{rawhtml}
\begin{rawhtml}

<!-- l2h_begin sing_l2h 42 -->
\end{rawhtml}
$K=Z/7[a]$
\begin{rawhtml}
<!-- l2h_end sing_l2h 42 -->

\end{rawhtml}
\begin{rawhtml}

<!-- l2h_begin sing_l2h 43 -->
\end{rawhtml}
$a.$
\begin{rawhtml}
<!-- l2h_end sing_l2h 43 -->

\end{rawhtml}
\begin{rawhtml}

<!-- l2h_begin sing_l2h 44 -->
\end{rawhtml}
$K$
\begin{rawhtml}
<!-- l2h_end sing_l2h 44 -->

\end{rawhtml}
\begin{rawhtml}

<!-- l2h_begin sing_l2h 45 -->
\end{rawhtml}
$\mu_a=a^2+a+3$,
\begin{rawhtml}
<!-- l2h_end sing_l2h 45 -->

\end{rawhtml}
\begin{rawhtml}

<!-- l2h_begin sing_l2h 46 -->
\end{rawhtml}
$R[x,y,z]$
\begin{rawhtml}
<!-- l2h_end sing_l2h 46 -->

\end{rawhtml}
\begin{rawhtml}

<!-- l2h_begin sing_l2h 47 -->
\end{rawhtml}
$R$
\begin{rawhtml}
<!-- l2h_end sing_l2h 47 -->

\end{rawhtml}
\begin{rawhtml}

<!-- l2h_begin sing_l2h 48 -->
\end{rawhtml}
$R(j)[x,y,z]$
\begin{rawhtml}
<!-- l2h_end sing_l2h 48 -->

\end{rawhtml}
\begin{rawhtml}

<!-- l2h_begin sing_l2h 49 -->
\end{rawhtml}
$j$
\begin{rawhtml}
<!-- l2h_end sing_l2h 49 -->

\end{rawhtml}
\begin{rawhtml}

<!-- l2h_begin sing_l2h 50 -->
\end{rawhtml}
$R(i)[x,y,z]$
\begin{rawhtml}
<!-- l2h_end sing_l2h 50 -->

\end{rawhtml}
\begin{rawhtml}

<!-- l2h_begin sing_l2h 51 -->
\end{rawhtml}
$i$
\begin{rawhtml}
<!-- l2h_end sing_l2h 51 -->

\end{rawhtml}
\begin{rawhtml}

<!-- l2h_begin sing_l2h 52 -->
\end{rawhtml}
$\hbox{GF}(p^n)$ with $p^n$ elements, where $p^n$ has to be smaller or equal $2^{15}$.
\begin{rawhtml}
<!-- l2h_end sing_l2h 52 -->

\end{rawhtml}
\begin{rawhtml}

<!-- l2h_begin sing_l2h 53 -->
\end{rawhtml}
$\hbox{GF}(p^n)$
\begin{rawhtml}
<!-- l2h_end sing_l2h 53 -->

\end{rawhtml}
\begin{rawhtml}

<!-- l2h_begin sing_l2h 54 -->
\end{rawhtml}
$K[x_1,\ldots,x_n]$,
\begin{rawhtml}
<!-- l2h_end sing_l2h 54 -->

\end{rawhtml}
\begin{rawhtml}

<!-- l2h_begin sing_l2h 55 -->
\end{rawhtml}
$\hbox{Loc}_{(x)}K[x_1,\ldots,x_n])$.
\begin{rawhtml}
<!-- l2h_end sing_l2h 55 -->

\end{rawhtml}
\begin{rawhtml}

<!-- l2h_begin sing_l2h 56 -->
\end{rawhtml}
$1 < x$
\begin{rawhtml}
<!-- l2h_end sing_l2h 56 -->

\end{rawhtml}
\begin{rawhtml}

<!-- l2h_begin sing_l2h 57 -->
\end{rawhtml}
$ m \times n $
\begin{rawhtml}
<!-- l2h_end sing_l2h 57 -->

\end{rawhtml}
\begin{rawhtml}

<!-- l2h_begin sing_l2h 58 -->
\end{rawhtml}
\quad
\begin{rawhtml}
<!-- l2h_end sing_l2h 58 -->

\end{rawhtml}
\begin{rawhtml}

<!-- l2h_begin sing_l2h 59 -->
\end{rawhtml}
$\alpha$
\begin{rawhtml}
<!-- l2h_end sing_l2h 59 -->

\end{rawhtml}
\begin{rawhtml}

<!-- l2h_begin sing_l2h 60 -->
\end{rawhtml}
$ i_{1,1} $
\begin{rawhtml}
<!-- l2h_end sing_l2h 60 -->

\end{rawhtml}
\begin{rawhtml}

<!-- l2h_begin sing_l2h 61 -->
\end{rawhtml}
$i+i+i$
\begin{rawhtml}
<!-- l2h_end sing_l2h 61 -->

\end{rawhtml}
\begin{rawhtml}

<!-- l2h_begin sing_l2h 62 -->
\end{rawhtml}
$\alpha, \beta$
\begin{rawhtml}
<!-- l2h_end sing_l2h 62 -->

\end{rawhtml}
\begin{rawhtml}

<!-- l2h_begin sing_l2h 63 -->
\end{rawhtml}
$i_1$
\begin{rawhtml}
<!-- l2h_end sing_l2h 63 -->

\end{rawhtml}
\begin{rawhtml}

<!-- l2h_begin sing_l2h 64 -->
\end{rawhtml}
$i_{1,1}$
\begin{rawhtml}
<!-- l2h_end sing_l2h 64 -->

\end{rawhtml}
\begin{rawhtml}

<!-- l2h_begin sing_l2h 65 -->
\end{rawhtml}
$(h1+h2)/h1 \cong h2/(h1 \cap h2)$
\begin{rawhtml}
<!-- l2h_end sing_l2h 65 -->

\end{rawhtml}
\begin{rawhtml}

<!-- l2h_begin sing_l2h 66 -->
\end{rawhtml}
$Q \rightarrow  Q(a, \ldots)$
\begin{rawhtml}
<!-- l2h_end sing_l2h 66 -->

\end{rawhtml}
\begin{rawhtml}

<!-- l2h_begin sing_l2h 67 -->
\end{rawhtml}
$Q \rightarrow R$
\begin{rawhtml}
<!-- l2h_end sing_l2h 67 -->

\end{rawhtml}
\begin{rawhtml}

<!-- l2h_begin sing_l2h 68 -->
\end{rawhtml}
$Q \rightarrow  C$
\begin{rawhtml}
<!-- l2h_end sing_l2h 68 -->

\end{rawhtml}
\begin{rawhtml}

<!-- l2h_begin sing_l2h 69 -->
\end{rawhtml}
$Z/p \rightarrow  (Z/p)(a, \ldots)$
\begin{rawhtml}
<!-- l2h_end sing_l2h 69 -->

\end{rawhtml}
\begin{rawhtml}

<!-- l2h_begin sing_l2h 70 -->
\end{rawhtml}
$Z/p \rightarrow  GF(p^n)$
\begin{rawhtml}
<!-- l2h_end sing_l2h 70 -->

\end{rawhtml}
\begin{rawhtml}

<!-- l2h_begin sing_l2h 71 -->
\end{rawhtml}
$Z/p \rightarrow  R$
\begin{rawhtml}
<!-- l2h_end sing_l2h 71 -->

\end{rawhtml}
\begin{rawhtml}

<!-- l2h_begin sing_l2h 72 -->
\end{rawhtml}
$R \rightarrow C$
\begin{rawhtml}
<!-- l2h_end sing_l2h 72 -->

\end{rawhtml}
\begin{rawhtml}

<!-- l2h_begin sing_l2h 73 -->
\end{rawhtml}
% This is quite a hack, but for now it works.
$Z/p \rightarrow Q,
\quad
[i]_p \mapsto i \in [-p/2, \, p/2]
\subseteq Z$
\begin{rawhtml}
<!-- l2h_end sing_l2h 73 -->

\end{rawhtml}
\begin{rawhtml}

<!-- l2h_begin sing_l2h 74 -->
\end{rawhtml}
$Z/p \rightarrow Z/p^\prime,
\quad
[i]_p \mapsto i \in [-p/2, \, p/2] \subseteq Z, \;
i \mapsto [i]_{p^\prime} \in Z/p^\prime$
\begin{rawhtml}
<!-- l2h_end sing_l2h 74 -->

\end{rawhtml}
\begin{rawhtml}

<!-- l2h_begin sing_l2h 75 -->
\end{rawhtml}
$C \rightarrow R, \quad$ the real part
\begin{rawhtml}
<!-- l2h_end sing_l2h 75 -->

\end{rawhtml}
\begin{rawhtml}

<!-- l2h_begin sing_l2h 76 -->
\end{rawhtml}
$Q \rightarrow Z/p$
\begin{rawhtml}
<!-- l2h_end sing_l2h 76 -->

\end{rawhtml}
\begin{rawhtml}

<!-- l2h_begin sing_l2h 77 -->
\end{rawhtml}
$Q \rightarrow (Z/p)(a, \ldots)$
\begin{rawhtml}
<!-- l2h_end sing_l2h 77 -->

\end{rawhtml}
\begin{rawhtml}

<!-- l2h_begin sing_l2h 78 -->
\end{rawhtml}
$M$
\begin{rawhtml}
<!-- l2h_end sing_l2h 78 -->

\end{rawhtml}
\begin{rawhtml}

<!-- l2h_begin sing_l2h 79 -->
\end{rawhtml}
$R^n$,
\begin{rawhtml}
<!-- l2h_end sing_l2h 79 -->

\end{rawhtml}
\begin{rawhtml}

<!-- l2h_begin sing_l2h 80 -->
\end{rawhtml}
$v_1, \ldots, v_k$, then $v_1, \ldots, v_k$
\begin{rawhtml}
<!-- l2h_end sing_l2h 80 -->

\end{rawhtml}
\begin{rawhtml}

<!-- l2h_begin sing_l2h 81 -->
\end{rawhtml}
$R^n/M$
\begin{rawhtml}
<!-- l2h_end sing_l2h 81 -->

\end{rawhtml}
\begin{rawhtml}

<!-- l2h_begin sing_l2h 82 -->
\end{rawhtml}
n$\times$k
\begin{rawhtml}
<!-- l2h_end sing_l2h 82 -->

\end{rawhtml}
\begin{rawhtml}

<!-- l2h_begin sing_l2h 83 -->
\end{rawhtml}
R$^n$/M,
\begin{rawhtml}
<!-- l2h_end sing_l2h 83 -->

\end{rawhtml}
\begin{rawhtml}

<!-- l2h_begin sing_l2h 84 -->
\end{rawhtml}
$v_1, \ldots, v_k$
\begin{rawhtml}
<!-- l2h_end sing_l2h 84 -->

\end{rawhtml}
\begin{rawhtml}

<!-- l2h_begin sing_l2h 85 -->
\end{rawhtml}
$(h1+h2)/h1=h2/(h1 \cap h2)$
\begin{rawhtml}
<!-- l2h_end sing_l2h 85 -->

\end{rawhtml}
\begin{rawhtml}

<!-- l2h_begin sing_l2h 86 -->
\end{rawhtml}
$k[X,Y]$
\begin{rawhtml}
<!-- l2h_end sing_l2h 86 -->

\end{rawhtml}
\begin{rawhtml}

<!-- l2h_begin sing_l2h 87 -->
\end{rawhtml}
$k_1[X]$
\begin{rawhtml}
<!-- l2h_end sing_l2h 87 -->

\end{rawhtml}
\begin{rawhtml}

<!-- l2h_begin sing_l2h 88 -->
\end{rawhtml}
$k_2[Y]$
\begin{rawhtml}
<!-- l2h_end sing_l2h 88 -->

\end{rawhtml}
\begin{rawhtml}

<!-- l2h_begin sing_l2h 89 -->
\end{rawhtml}
$k_1$
\begin{rawhtml}
<!-- l2h_end sing_l2h 89 -->

\end{rawhtml}
\begin{rawhtml}

<!-- l2h_begin sing_l2h 90 -->
\end{rawhtml}
$k_2$
\begin{rawhtml}
<!-- l2h_end sing_l2h 90 -->

\end{rawhtml}
\begin{rawhtml}

<!-- l2h_begin sing_l2h 91 -->
\end{rawhtml}
$C$
\begin{rawhtml}
<!-- l2h_end sing_l2h 91 -->

\end{rawhtml}
\begin{rawhtml}

<!-- l2h_begin sing_l2h 92 -->
\end{rawhtml}
$R^n/M$, if $R$ denotes the basering and
$M$ a homogeneous submodule of $R^n$ and the argument represents a
resolution of
$R^n/M$.
\begin{rawhtml}
<!-- l2h_end sing_l2h 92 -->

\end{rawhtml}
\begin{rawhtml}

<!-- l2h_begin sing_l2h 93 -->
\end{rawhtml}
The entry d of the intmat at place (i,j) is the minimal number of
generators in degree i+j of the j-th syzygy module (= module of
relations) of $R^n/M$ (the 0th (resp.\ 1st) syzygy module of $R^n/M$ is
$R^n$ (resp.\ $M$)).
\begin{rawhtml}
<!-- l2h_end sing_l2h 93 -->

\end{rawhtml}
\begin{rawhtml}

<!-- l2h_begin sing_l2h 94 -->
\end{rawhtml}
$$
0 \longleftarrow r/j \longleftarrow r(1)
\buildrel{T[1]}\over{\longleftarrow} r(2) \oplus r^3(3)
\buildrel{T[2]}\over{\longleftarrow} r^4(4)
\buildrel{T[3]}\over{\longleftarrow} r(5)
\longleftarrow 0 \quad .
$$
\begin{rawhtml}
<!-- l2h_end sing_l2h 94 -->

\end{rawhtml}
\begin{rawhtml}

<!-- l2h_begin sing_l2h 95 -->
\end{rawhtml}
The third argument is used to return the matrix T of coefficients
such that {\tt matrix}(J) = T*M.
\begin{rawhtml}
<!-- l2h_end sing_l2h 95 -->

\end{rawhtml}
\begin{rawhtml}

<!-- l2h_begin sing_l2h 96 -->
\end{rawhtml}
$M=(m_{ij})$
\begin{rawhtml}
<!-- l2h_end sing_l2h 96 -->

\end{rawhtml}
\begin{rawhtml}

<!-- l2h_begin sing_l2h 97 -->
\end{rawhtml}
$$J_j = z^0 \cdot m_{1j} + z^1 \cdot m_{2j} + ... + z^{d-1} \cdot m_{dj},$$
while for a module J the i-th component of the j-th generator is
equal to the entry [i,j] of {\tt matrix}(J), and we get
\begin{rawhtml}
<!-- l2h_end sing_l2h 97 -->

\end{rawhtml}
\begin{rawhtml}

<!-- l2h_begin sing_l2h 98 -->
\end{rawhtml}
$$ J_{i,j} = z^0 \cdot m_{(i-1)d+1,j} + z^1 \cdot m_{(i-1)d+2,j} + ... +
z^{d-1} \cdot m_{id,j}.$$
\begin{rawhtml}
<!-- l2h_end sing_l2h 98 -->

\end{rawhtml}
\begin{rawhtml}

<!-- l2h_begin sing_l2h 99 -->
\end{rawhtml}
$${\rm contract}(x^A ,  x^B) := \cases{ x^{(B-A)}, &if $B\ge A$
componentwise\cr 0,&otherwise.\cr}$$
\begin{rawhtml}
<!-- l2h_end sing_l2h 99 -->

\end{rawhtml}
\begin{rawhtml}

<!-- l2h_begin sing_l2h 100 -->
\end{rawhtml}
$x$.
\begin{rawhtml}
<!-- l2h_end sing_l2h 100 -->

\end{rawhtml}
\begin{rawhtml}

<!-- l2h_begin sing_l2h 101 -->
\end{rawhtml}
$x_i$
\begin{rawhtml}
<!-- l2h_end sing_l2h 101 -->

\end{rawhtml}
\begin{rawhtml}

<!-- l2h_begin sing_l2h 102 -->
\end{rawhtml}
$x_i>1$ then $x_i$
\begin{rawhtml}
<!-- l2h_end sing_l2h 102 -->

\end{rawhtml}
\begin{rawhtml}

<!-- l2h_begin sing_l2h 103 -->
\end{rawhtml}
$f_1,\dots,f_k$ of I, let $f'_i$ be obtained from
$f_i$ by deleting the terms divisible by $x_i\cdot m$ for all i with $x_i<1$.
Then $f'_1,\dots,f'_k$ generate I.
\begin{rawhtml}
<!-- l2h_end sing_l2h 103 -->

\end{rawhtml}
\begin{rawhtml}

<!-- l2h_begin sing_l2h 104 -->
\end{rawhtml}
$$...\longrightarrow F_2 \buildrel{A_2}\over{\longrightarrow} F_1
\buildrel{A_1}\over{\longrightarrow} R\longrightarrow R/I
\longrightarrow 0.$$
\begin{rawhtml}
<!-- l2h_end sing_l2h 104 -->

\end{rawhtml}
\begin{rawhtml}

<!-- l2h_begin sing_l2h 105 -->
\end{rawhtml}
$M_i={\tt module} (A_i)$, i=1..k.
\begin{rawhtml}
<!-- l2h_end sing_l2h 105 -->

\end{rawhtml}
\begin{rawhtml}

<!-- l2h_begin sing_l2h 106 -->
\end{rawhtml}
${\tt L[i]}=M_i$
\begin{rawhtml}
<!-- l2h_end sing_l2h 106 -->

\end{rawhtml}
\begin{rawhtml}

<!-- l2h_begin sing_l2h 107 -->
\end{rawhtml}
$I \cap K[U]=(0)$,
\begin{rawhtml}
<!-- l2h_end sing_l2h 107 -->

\end{rawhtml}
\begin{rawhtml}

<!-- l2h_begin sing_l2h 108 -->
\end{rawhtml}
input: $f_1,\dots,f_n$
\begin{rawhtml}
<!-- l2h_end sing_l2h 108 -->

\end{rawhtml}
\begin{rawhtml}

<!-- l2h_begin sing_l2h 109 -->
\end{rawhtml}
output: $g_1,\dots,g_s$ with $s \leq n$ and the properties
\begin{rawhtml}
<!-- l2h_end sing_l2h 109 -->

\end{rawhtml}
\begin{rawhtml}

<!-- l2h_begin sing_l2h 110 -->
\end{rawhtml}
$(f_1,\dots,f_n) = (g_1,\dots,g_s)$
\begin{rawhtml}
<!-- l2h_end sing_l2h 110 -->

\end{rawhtml}
\begin{rawhtml}

<!-- l2h_begin sing_l2h 111 -->
\end{rawhtml}
$L(g_i)\neq L(g_j)$ for all $i\neq j$
\begin{rawhtml}
<!-- l2h_end sing_l2h 111 -->

\end{rawhtml}
\begin{rawhtml}

<!-- l2h_begin sing_l2h 112 -->
\end{rawhtml}
$L(g_i)$
\begin{rawhtml}
<!-- l2h_end sing_l2h 112 -->

\end{rawhtml}
\begin{rawhtml}

<!-- l2h_begin sing_l2h 113 -->
\end{rawhtml}
$\{g_1,\dots,g_{i-1},g_{i+1},\dots,g_s\}$
\begin{rawhtml}
<!-- l2h_end sing_l2h 113 -->

\end{rawhtml}
\begin{rawhtml}

<!-- l2h_begin sing_l2h 114 -->
\end{rawhtml}
$L(g_i) | L(g_j)$ for any $i \neq j$,
\begin{rawhtml}
<!-- l2h_end sing_l2h 114 -->

\end{rawhtml}
\begin{rawhtml}

<!-- l2h_begin sing_l2h 115 -->
\end{rawhtml}
$ecart(g_i) > ecart(g_j)$
\begin{rawhtml}
<!-- l2h_end sing_l2h 115 -->

\end{rawhtml}
\begin{rawhtml}

<!-- l2h_begin sing_l2h 116 -->
\end{rawhtml}
Here, $L(g)$ denotes the leading term of $g$ and
$ecart(g):=deg(g)-deg(L(g))$.
\begin{rawhtml}
<!-- l2h_end sing_l2h 116 -->

\end{rawhtml}
\begin{rawhtml}

<!-- l2h_begin sing_l2h 117 -->
\end{rawhtml}
$M_i={\tt module}(A_i)$, i=1..k.
\begin{rawhtml}
<!-- l2h_end sing_l2h 117 -->

\end{rawhtml}
\begin{rawhtml}

<!-- l2h_begin sing_l2h 118 -->
\end{rawhtml}
represents $h_1/(h_1 \cap h_2) \cong (h_1+h_2)/h_2$
\begin{rawhtml}
<!-- l2h_end sing_l2h 118 -->

\end{rawhtml}
\begin{rawhtml}

<!-- l2h_begin sing_l2h 119 -->
\end{rawhtml}
$h_1$ and $h_2$
\begin{rawhtml}
<!-- l2h_end sing_l2h 119 -->

\end{rawhtml}
\begin{rawhtml}

<!-- l2h_begin sing_l2h 120 -->
\end{rawhtml}
$R^l$
\begin{rawhtml}
<!-- l2h_end sing_l2h 120 -->

\end{rawhtml}
\begin{rawhtml}

<!-- l2h_begin sing_l2h 121 -->
\end{rawhtml}
$H_1$, resp.\ $H_2$,
\begin{rawhtml}
<!-- l2h_end sing_l2h 121 -->

\end{rawhtml}
\begin{rawhtml}

<!-- l2h_begin sing_l2h 122 -->
\end{rawhtml}
be the matrices of size $l \times k$, resp.\ $l \times m$, having the
generators of $h_1$, resp.\ $h_2$,
\begin{rawhtml}
<!-- l2h_end sing_l2h 122 -->

\end{rawhtml}
\begin{rawhtml}

<!-- l2h_begin sing_l2h 123 -->
\end{rawhtml}
$h_1/(h_1 \cap h_2) \cong R^k / ker(\overline{H_1})$
\begin{rawhtml}
<!-- l2h_end sing_l2h 123 -->

\end{rawhtml}
\begin{rawhtml}

<!-- l2h_begin sing_l2h 124 -->
\end{rawhtml}
$\overline{H_1}: R^k \rightarrow R^l/Im(H_2)=R^l/h_2$
is the induced map.
\begin{rawhtml}
<!-- l2h_end sing_l2h 124 -->

\end{rawhtml}
\begin{rawhtml}

<!-- l2h_begin sing_l2h 125 -->
\end{rawhtml}
$coker(A)=F_0/M$
$$...\longrightarrow F_2 \buildrel{A_2}\over{\longrightarrow} F_1
\buildrel{A_1}\over{\longrightarrow} F_0\longrightarrow F_0/M
\longrightarrow 0,$$
\begin{rawhtml}
<!-- l2h_end sing_l2h 125 -->

\end{rawhtml}
\begin{rawhtml}

<!-- l2h_begin sing_l2h 126 -->
\end{rawhtml}
$A_1$
\begin{rawhtml}
<!-- l2h_end sing_l2h 126 -->

\end{rawhtml}
\begin{rawhtml}

<!-- l2h_begin sing_l2h 127 -->
\end{rawhtml}
$M_i={\tt module}(A_i)$, i=1...k.
\begin{rawhtml}
<!-- l2h_end sing_l2h 127 -->

\end{rawhtml}
\begin{rawhtml}

<!-- l2h_begin sing_l2h 128 -->
\end{rawhtml}
${\tt L[i]}\neq 0$ for $i \le p$,
\begin{rawhtml}
<!-- l2h_end sing_l2h 128 -->

\end{rawhtml}
\begin{rawhtml}

<!-- l2h_begin sing_l2h 129 -->
\end{rawhtml}
$A_1$=matrix(M),
\begin{rawhtml}
<!-- l2h_end sing_l2h 129 -->

\end{rawhtml}
\begin{rawhtml}

<!-- l2h_begin sing_l2h 130 -->
\end{rawhtml}
$coker(A_1)=F_0/M$
$$...\longrightarrow F_2 \buildrel{A_2}\over{\longrightarrow} F_1 \buildrel{A_1}\over{\longrightarrow} F_0\longrightarrow F_0/M\longrightarrow 0,$$
\begin{rawhtml}
<!-- l2h_end sing_l2h 130 -->

\end{rawhtml}
\begin{rawhtml}

<!-- l2h_begin sing_l2h 131 -->
\end{rawhtml}
(${\tt L[i]}=M_i$
\begin{rawhtml}
<!-- l2h_end sing_l2h 131 -->

\end{rawhtml}
\begin{rawhtml}

<!-- l2h_begin sing_l2h 132 -->
\end{rawhtml}
$d$ in row $i$ and column $j$
\begin{rawhtml}
<!-- l2h_end sing_l2h 132 -->

\end{rawhtml}
\begin{rawhtml}

<!-- l2h_begin sing_l2h 133 -->
\end{rawhtml}
$i+j$ of the $j$-th
\begin{rawhtml}
<!-- l2h_end sing_l2h 133 -->

\end{rawhtml}
\begin{rawhtml}

<!-- l2h_begin sing_l2h 134 -->
\end{rawhtml}
$R^n/M$ (the 0th and 1st syzygy module of $R^n/M$ is $R^n$ and $M$, resp.).
\begin{rawhtml}
<!-- l2h_end sing_l2h 134 -->

\end{rawhtml}
\begin{rawhtml}

<!-- l2h_begin sing_l2h 135 -->
\end{rawhtml}
${\tt R}^n$.
\begin{rawhtml}
<!-- l2h_end sing_l2h 135 -->

\end{rawhtml}
\begin{rawhtml}

<!-- l2h_begin sing_l2h 136 -->
\end{rawhtml}
$\{a \in R \mid aJ \subset I\}$,
\begin{rawhtml}
<!-- l2h_end sing_l2h 136 -->

\end{rawhtml}
\begin{rawhtml}

<!-- l2h_begin sing_l2h 137 -->
\end{rawhtml}
$\{b \in R^n \mid bJ \subset M\}$.
\begin{rawhtml}
<!-- l2h_end sing_l2h 137 -->

\end{rawhtml}
\begin{rawhtml}

<!-- l2h_begin sing_l2h 138 -->
\end{rawhtml}
\noindent
Let $0 \rightarrow\ \bigoplus_a K[x]e_{a,n}\ \rightarrow\ \dots
  \rightarrow\ \bigoplus_a K[x]e_{a,0}\ \rightarrow\
  I\ \rightarrow\ 0$
be a minimal resolution of I considered with homogeneous maps of degree 0.
The regularity is the smallest number $s$ with the property deg($e_{a,i})
 \leq s+i$ for all $i$.
\begin{rawhtml}
<!-- l2h_end sing_l2h 138 -->

\end{rawhtml}
\begin{rawhtml}

<!-- l2h_begin sing_l2h 139 -->
\end{rawhtml}
computes the permutation {\tt v}
which orders the ideal, resp.\ module, {\tt I} by its initial terms,
starting with the smallest, that is, {\tt I(v[i]) < I(v[i+1])} for all
{\tt i}.
\begin{rawhtml}
<!-- l2h_end sing_l2h 139 -->

\end{rawhtml}
\begin{rawhtml}

<!-- l2h_begin sing_l2h 140 -->
\end{rawhtml}
$A_1={\tt matrix}(M)$.
\begin{rawhtml}
<!-- l2h_end sing_l2h 140 -->

\end{rawhtml}
\begin{rawhtml}

<!-- l2h_begin sing_l2h 141 -->
\end{rawhtml}
$coker(A_1)=F_0/M$
$$...\longrightarrow F_2 \buildrel{A_2}\over{\longrightarrow} F_1 \buildrel{A_1}\over{\longrightarrow} F_0\longrightarrow F_0/M\longrightarrow 0.$$
\begin{rawhtml}
<!-- l2h_end sing_l2h 141 -->

\end{rawhtml}
\begin{rawhtml}

<!-- l2h_begin sing_l2h 142 -->
\end{rawhtml}
{\tt vandermonde(p,v,d)} computes the (unique) polynomial of degree
@code{d} with prescribed values {\tt v[1],...,v[N]} at the points
{\tt p}$^0,\dots,$ {\tt p}$^{N-1}$, {\tt N=(d+1)}$^n$, $n$ the
number of ring variables.

The returned polynomial is $\sum
c_{\alpha_1\ldots\alpha_n}\cdot x_1^{\alpha_1} \cdot \dots \cdot
x_n^{\alpha_n}$, where the coefficients
$c_{\alpha_1\ldots\alpha_n}$ are the solution of the (transposed)
Vandermonde system of linear equations
$$ \sum_{\alpha_1+\ldots+\alpha_n\leq d} c_{\alpha_1\ldots\alpha_n} \cdot
{\tt p}_1^{(k-1)\alpha_1}\cdot\dots\cdot {\tt p}_n^{(k-1)\alpha_n} =
{\tt v}[k], \quad  k=1,\dots,{\tt N}.$$
\begin{rawhtml}
<!-- l2h_end sing_l2h 142 -->

\end{rawhtml}
\begin{rawhtml}

<!-- l2h_begin sing_l2h 143 -->
\end{rawhtml}
the ground field has to be the field of rational
numbers. Moreover, {\tt ncols(p)==}$n$, the number of variables in the
basering, and all the given generators have to be numbers different from
0,1 or -1. Finally, {\tt ncols(v)==(d+1)$^n$}, and all given generators have
to be numbers.
\begin{rawhtml}
<!-- l2h_end sing_l2h 143 -->

\end{rawhtml}
\begin{rawhtml}

<!-- l2h_begin sing_l2h 144 -->
\end{rawhtml}
$K[[x_1,\ldots,x_n]]$
\begin{rawhtml}
<!-- l2h_end sing_l2h 144 -->

\end{rawhtml}
\begin{rawhtml}

<!-- l2h_begin sing_l2h 145 -->
\end{rawhtml}
$$
\hbox{milnor}(f) = \hbox{dim}_K(K[[x_1,\ldots,x_n]]/\hbox{jacob}(f)),
$$
respectively
$$
\hbox{tjurina}(f) = \hbox{dim}_K(K[[x_1,\ldots,x_n]]/((f)+\hbox{jacob}(f)))
$$
where
\begin{rawhtml}
<!-- l2h_end sing_l2h 145 -->

\end{rawhtml}
\begin{rawhtml}

<!-- l2h_begin sing_l2h 146 -->
\end{rawhtml}
$\hbox{Loc}_{(x)}K[x_1,\ldots,x_n]$,
\begin{rawhtml}
<!-- l2h_end sing_l2h 146 -->

\end{rawhtml}
\begin{rawhtml}

<!-- l2h_begin sing_l2h 147 -->
\end{rawhtml}
$K[x_1,\ldots,x_n]$
\begin{rawhtml}
<!-- l2h_end sing_l2h 147 -->

\end{rawhtml}
\begin{rawhtml}

<!-- l2h_begin sing_l2h 148 -->
\end{rawhtml}
$(x_1,\ldots,x_n)$.
\begin{rawhtml}
<!-- l2h_end sing_l2h 148 -->

\end{rawhtml}
\begin{rawhtml}

<!-- l2h_begin sing_l2h 149 -->
\end{rawhtml}
$\hbox{Loc}_{(x)}K[x_1,\ldots,x_n])$
\begin{rawhtml}
<!-- l2h_end sing_l2h 149 -->

\end{rawhtml}
\begin{rawhtml}

<!-- l2h_begin sing_l2h 150 -->
\end{rawhtml}
$\hbox{dim}_K(\hbox{Loc}_{(x,y,z)}K[x,y,z]/\hbox{jacob}(f))$
\begin{rawhtml}
<!-- l2h_end sing_l2h 150 -->

\end{rawhtml}
\begin{rawhtml}

<!-- l2h_begin sing_l2h 151 -->
\end{rawhtml}
$\hbox{dim}_K(K[x,y,z]/\hbox{jacob}(f))$
\begin{rawhtml}
<!-- l2h_end sing_l2h 151 -->

\end{rawhtml}
\begin{rawhtml}

<!-- l2h_begin sing_l2h 152 -->
\end{rawhtml}
$\hbox{dim}_K(\hbox{Loc}_{(x,y,z)}K[x,y,z]/(\hbox{jacob}(f)+(f)))$
\begin{rawhtml}
<!-- l2h_end sing_l2h 152 -->

\end{rawhtml}
\begin{rawhtml}

<!-- l2h_begin sing_l2h 153 -->
\end{rawhtml}
$\hbox{dim}_K(K[x,y,z]/(\hbox{jacob}(f)+(f)))$,
\begin{rawhtml}
<!-- l2h_end sing_l2h 153 -->

\end{rawhtml}
\begin{rawhtml}

<!-- l2h_begin sing_l2h 154 -->
\end{rawhtml}
$j+(f)$
\begin{rawhtml}
<!-- l2h_end sing_l2h 154 -->

\end{rawhtml}
\begin{rawhtml}

<!-- l2h_begin sing_l2h 155 -->
\end{rawhtml}
$f=0$
\begin{rawhtml}
<!-- l2h_end sing_l2h 155 -->

\end{rawhtml}
\begin{rawhtml}

<!-- l2h_begin sing_l2h 156 -->
\end{rawhtml}
$m$
\begin{rawhtml}
<!-- l2h_end sing_l2h 156 -->

\end{rawhtml}
\begin{rawhtml}

<!-- l2h_begin sing_l2h 157 -->
\end{rawhtml}
$i, j$
\begin{rawhtml}
<!-- l2h_end sing_l2h 157 -->

\end{rawhtml}
\begin{rawhtml}

<!-- l2h_begin sing_l2h 158 -->
\end{rawhtml}
$$
\hbox{sat}(i,j)=\{x\in R\;|\; \exists\;n\hbox{ s.t. }
x\cdot(j^n)\subseteq i\}
= \bigcup_{n=1}^\infty i:j^n$$
\begin{rawhtml}
<!-- l2h_end sing_l2h 158 -->

\end{rawhtml}
\begin{rawhtml}

<!-- l2h_begin sing_l2h 159 -->
\end{rawhtml}
$sat(j+(f),m)$
\begin{rawhtml}
<!-- l2h_end sing_l2h 159 -->

\end{rawhtml}
\begin{rawhtml}

<!-- l2h_begin sing_l2h 160 -->
\end{rawhtml}
$dim_{Q(t)}Q(t)[x,y]/j$.
\begin{rawhtml}
<!-- l2h_end sing_l2h 160 -->

\end{rawhtml}
\begin{rawhtml}

<!-- l2h_begin sing_l2h 161 -->
\end{rawhtml}
$a \in Q$
\begin{rawhtml}
<!-- l2h_end sing_l2h 161 -->

\end{rawhtml}
\begin{rawhtml}

<!-- l2h_begin sing_l2h 162 -->
\end{rawhtml}
$dim_Q Q[x,y]/j_0$,
\begin{rawhtml}
<!-- l2h_end sing_l2h 162 -->

\end{rawhtml}
\begin{rawhtml}

<!-- l2h_begin sing_l2h 163 -->
\end{rawhtml}
$j_0=j|_{t=a}$.
\begin{rawhtml}
<!-- l2h_end sing_l2h 163 -->

\end{rawhtml}
\begin{rawhtml}

<!-- l2h_begin sing_l2h 164 -->
\end{rawhtml}
$T^1$
\begin{rawhtml}
<!-- l2h_end sing_l2h 164 -->

\end{rawhtml}
\begin{rawhtml}

<!-- l2h_begin sing_l2h 165 -->
\end{rawhtml}
$T^2$
\begin{rawhtml}
<!-- l2h_end sing_l2h 165 -->

\end{rawhtml}
\begin{rawhtml}

<!-- l2h_begin sing_l2h 166 -->
\end{rawhtml}
$h_1,\ldots,h_r$
\begin{rawhtml}
<!-- l2h_end sing_l2h 166 -->

\end{rawhtml}
\begin{rawhtml}

<!-- l2h_begin sing_l2h 167 -->
\end{rawhtml}
$I \subset R$ is the ideal generated by $f_1,...,f_s$, then any infinitesimal
deformation of $R/I$ over $K[\varepsilon]/(\varepsilon^2)$ is given
by $f+\varepsilon g$,
where $f=(f_1,...,f_s)$, $g$ a $K$-linear combination of the $h_i$.
\begin{rawhtml}
<!-- l2h_end sing_l2h 167 -->

\end{rawhtml}
\begin{rawhtml}

<!-- l2h_begin sing_l2h 168 -->
\end{rawhtml}
$d$
\begin{rawhtml}
<!-- l2h_end sing_l2h 168 -->

\end{rawhtml}
\begin{rawhtml}

<!-- l2h_begin sing_l2h 169 -->
\end{rawhtml}
$p^k$
\begin{rawhtml}
<!-- l2h_end sing_l2h 169 -->

\end{rawhtml}
\begin{rawhtml}

<!-- l2h_begin sing_l2h 170 -->
\end{rawhtml}
$f=(f_1,\ldots,f_n):k^r\rightarrow k^n$
\begin{rawhtml}
<!-- l2h_end sing_l2h 170 -->

\end{rawhtml}
\begin{rawhtml}

<!-- l2h_begin sing_l2h 171 -->
\end{rawhtml}
$$
\displaylines{
j=J \cap k[x_1,\ldots,x_n], \;\quad\hbox{\rm where}\cr
J=(x_1-f_1(t_1,\ldots,t_r),\ldots,x_n-f_n(t_1,\ldots,t_r))\subseteq
k[t_1,\ldots,t_r,x_1,\ldots,x_n]
}
$$
\begin{rawhtml}
<!-- l2h_end sing_l2h 171 -->

\end{rawhtml}
\begin{rawhtml}

<!-- l2h_begin sing_l2h 172 -->
\end{rawhtml}
$t_1,\ldots,t_r$.
\begin{rawhtml}
<!-- l2h_end sing_l2h 172 -->

\end{rawhtml}
\begin{rawhtml}

<!-- l2h_begin sing_l2h 173 -->
\end{rawhtml}
$f:(k^r,0)\rightarrow(k^n,0)$,
\begin{rawhtml}
<!-- l2h_end sing_l2h 173 -->

\end{rawhtml}
\begin{rawhtml}

<!-- l2h_begin sing_l2h 174 -->
\end{rawhtml}
$k^r\times(k^n,0)$,
\begin{rawhtml}
<!-- l2h_end sing_l2h 174 -->

\end{rawhtml}
\begin{rawhtml}

<!-- l2h_begin sing_l2h 175 -->
\end{rawhtml}
$$\hbox{pr}:k^r\times(k^n,0)\rightarrow(k^n,0)$$
can be computed.
\begin{rawhtml}
<!-- l2h_end sing_l2h 175 -->

\end{rawhtml}
\begin{rawhtml}

<!-- l2h_begin sing_l2h 176 -->
\end{rawhtml}
$\hbox{T}[7]^\prime$
\begin{rawhtml}
<!-- l2h_end sing_l2h 176 -->

\end{rawhtml}
\begin{rawhtml}

<!-- l2h_begin sing_l2h 177 -->
\end{rawhtml}
$P^4$.
\begin{rawhtml}
<!-- l2h_end sing_l2h 177 -->

\end{rawhtml}
\begin{rawhtml}

<!-- l2h_begin sing_l2h 178 -->
\end{rawhtml}
$K^3$
\begin{rawhtml}
<!-- l2h_end sing_l2h 178 -->

\end{rawhtml}
\begin{rawhtml}

<!-- l2h_begin sing_l2h 179 -->
\end{rawhtml}
$K^3$.
\begin{rawhtml}
<!-- l2h_end sing_l2h 179 -->

\end{rawhtml}
\begin{rawhtml}

<!-- l2h_begin sing_l2h 180 -->
\end{rawhtml}
$P^4$
\begin{rawhtml}
<!-- l2h_end sing_l2h 180 -->

\end{rawhtml}
\begin{rawhtml}

<!-- l2h_begin sing_l2h 181 -->
\end{rawhtml}
$\hbox{\rm ker} / \hbox{\rm Im}$
\begin{rawhtml}
<!-- l2h_end sing_l2h 181 -->

\end{rawhtml}
\begin{rawhtml}

<!-- l2h_begin sing_l2h 182 -->
\end{rawhtml}
$(n-1)$
\begin{rawhtml}
<!-- l2h_end sing_l2h 182 -->

\end{rawhtml}
\begin{rawhtml}

<!-- l2h_begin sing_l2h 183 -->
\end{rawhtml}
$A/(A \cap B)$
\begin{rawhtml}
<!-- l2h_end sing_l2h 183 -->

\end{rawhtml}
\begin{rawhtml}

<!-- l2h_begin sing_l2h 184 -->
\end{rawhtml}
$\hbox{Ext}^1(M,M)$, resp.\ $\hbox{Ext}^2(M,M)$,
\begin{rawhtml}
<!-- l2h_end sing_l2h 184 -->

\end{rawhtml}
\begin{rawhtml}

<!-- l2h_begin sing_l2h 185 -->
\end{rawhtml}
$\hbox{Ext}^1$
\begin{rawhtml}
<!-- l2h_end sing_l2h 185 -->

\end{rawhtml}
\begin{rawhtml}

<!-- l2h_begin sing_l2h 186 -->
\end{rawhtml}
$\hbox{Ext}^1$, $\hbox{Ext}^2$
\begin{rawhtml}
<!-- l2h_end sing_l2h 186 -->

\end{rawhtml}
\begin{rawhtml}

<!-- l2h_begin sing_l2h 187 -->
\end{rawhtml}
$\hbox{Ext}^k(R/J,R)$
\begin{rawhtml}
<!-- l2h_end sing_l2h 187 -->

\end{rawhtml}
\begin{rawhtml}

<!-- l2h_begin sing_l2h 188 -->
\end{rawhtml}
$J$
\begin{rawhtml}
<!-- l2h_end sing_l2h 188 -->

\end{rawhtml}
\begin{rawhtml}

<!-- l2h_begin sing_l2h 189 -->
\end{rawhtml}
($=\hbox{Ext}^1(K,K)$)
\begin{rawhtml}
<!-- l2h_end sing_l2h 189 -->

\end{rawhtml}
\begin{rawhtml}

<!-- l2h_begin sing_l2h 190 -->
\end{rawhtml}
($=\hbox{Ext}^2(K,K)$)
\begin{rawhtml}
<!-- l2h_end sing_l2h 190 -->

\end{rawhtml}
\begin{rawhtml}

<!-- l2h_begin sing_l2h 191 -->
\end{rawhtml}
$K=R/m$
\begin{rawhtml}
<!-- l2h_end sing_l2h 191 -->

\end{rawhtml}
\begin{rawhtml}

<!-- l2h_begin sing_l2h 192 -->
\end{rawhtml}
$R=Loc_m K[x,y]/(x^2-y^3)$, $m=(x,y)$.
\begin{rawhtml}
<!-- l2h_end sing_l2h 192 -->

\end{rawhtml}
\begin{rawhtml}

<!-- l2h_begin sing_l2h 193 -->
\end{rawhtml}
$\hbox{Ext}^1(m,m)$
\begin{rawhtml}
<!-- l2h_end sing_l2h 193 -->

\end{rawhtml}
\begin{rawhtml}

<!-- l2h_begin sing_l2h 194 -->
\end{rawhtml}
$\hbox{Ext}^2(K,K)$.
\begin{rawhtml}
<!-- l2h_end sing_l2h 194 -->

\end{rawhtml}
\begin{rawhtml}

<!-- l2h_begin sing_l2h 195 -->
\end{rawhtml}
$\hbox{Ext}^k(R/i,R)$
\begin{rawhtml}
<!-- l2h_end sing_l2h 195 -->

\end{rawhtml}
\begin{rawhtml}

<!-- l2h_begin sing_l2h 196 -->
\end{rawhtml}
$f\in k[x_1,\ldots,x_n,t]$
\begin{rawhtml}
<!-- l2h_end sing_l2h 196 -->

\end{rawhtml}
\begin{rawhtml}

<!-- l2h_begin sing_l2h 197 -->
\end{rawhtml}
$t$
\begin{rawhtml}
<!-- l2h_end sing_l2h 197 -->

\end{rawhtml}
\begin{rawhtml}

<!-- l2h_begin sing_l2h 198 -->
\end{rawhtml}
$V(\partial f/\partial x_1,\ldots,\partial f/\partial x_n) \setminus V(f)$
\begin{rawhtml}
<!-- l2h_end sing_l2h 198 -->

\end{rawhtml}
\begin{rawhtml}

<!-- l2h_begin sing_l2h 199 -->
\end{rawhtml}
$V(\partial f/\partial x_1,\ldots,\partial f/\partial x_n)$
\begin{rawhtml}
<!-- l2h_end sing_l2h 199 -->

\end{rawhtml}
\begin{rawhtml}

<!-- l2h_begin sing_l2h 200 -->
\end{rawhtml}
(in $P^n$),
\begin{rawhtml}
<!-- l2h_end sing_l2h 200 -->

\end{rawhtml}
\begin{rawhtml}

<!-- l2h_begin sing_l2h 201 -->
\end{rawhtml}
(in $k^n$)
\begin{rawhtml}
<!-- l2h_end sing_l2h 201 -->

\end{rawhtml}
\begin{rawhtml}

<!-- l2h_begin sing_l2h 202 -->
\end{rawhtml}
(in $(k^n,0)$),
\begin{rawhtml}
<!-- l2h_end sing_l2h 202 -->

\end{rawhtml}
\begin{rawhtml}

<!-- l2h_begin sing_l2h 203 -->
\end{rawhtml}
D$_k$(R)
\begin{rawhtml}
<!-- l2h_end sing_l2h 203 -->

\end{rawhtml}
\begin{rawhtml}

<!-- l2h_begin sing_l2h 204 -->
\end{rawhtml}
$(m+1)$
\begin{rawhtml}
<!-- l2h_end sing_l2h 204 -->

\end{rawhtml}
\begin{rawhtml}

<!-- l2h_begin sing_l2h 205 -->
\end{rawhtml}
$(n \times n)$-matrix.
\begin{rawhtml}
<!-- l2h_end sing_l2h 205 -->

\end{rawhtml}
\begin{rawhtml}

<!-- l2h_begin sing_l2h 206 -->
\end{rawhtml}
$\hbox{depth}(\hbox{D}_k(R))\geq m(m+1)/2 + m-1$
\begin{rawhtml}
<!-- l2h_end sing_l2h 206 -->

\end{rawhtml}
\begin{rawhtml}

<!-- l2h_begin sing_l2h 207 -->
\end{rawhtml}
$\sum x_i \partial /\partial x_{i+1}$.
\begin{rawhtml}
<!-- l2h_end sing_l2h 207 -->

\end{rawhtml}
\begin{rawhtml}

<!-- l2h_begin sing_l2h 208 -->
\end{rawhtml}
$f$
\begin{rawhtml}
<!-- l2h_end sing_l2h 208 -->

\end{rawhtml}
\begin{rawhtml}

<!-- l2h_begin sing_l2h 209 -->
\end{rawhtml}
$R/I$
\begin{rawhtml}
<!-- l2h_end sing_l2h 209 -->

\end{rawhtml}
\begin{rawhtml}

<!-- l2h_begin sing_l2h 210 -->
\end{rawhtml}
$A$
\begin{rawhtml}
<!-- l2h_end sing_l2h 210 -->

\end{rawhtml}
\begin{rawhtml}

<!-- l2h_begin sing_l2h 211 -->
\end{rawhtml}
$B$
\begin{rawhtml}
<!-- l2h_end sing_l2h 211 -->

\end{rawhtml}
\begin{rawhtml}

<!-- l2h_begin sing_l2h 212 -->
\end{rawhtml}
$m\times r$ and $m\times s$
\begin{rawhtml}
<!-- l2h_end sing_l2h 212 -->

\end{rawhtml}
\begin{rawhtml}

<!-- l2h_begin sing_l2h 213 -->
\end{rawhtml}
$$
R^r \buildrel{A}\over{\longrightarrow}
R^m \buildrel{B}\over{\longleftarrow} R^s\;.
$$
\begin{rawhtml}
<!-- l2h_end sing_l2h 213 -->

\end{rawhtml}
\begin{rawhtml}

<!-- l2h_begin sing_l2h 214 -->
\end{rawhtml}
$R^r \buildrel{A}\over{\longrightarrow}
R^m\longrightarrow
R^m/\hbox{Im}(B) \;.$
\begin{rawhtml}
<!-- l2h_end sing_l2h 214 -->

\end{rawhtml}
\begin{rawhtml}

<!-- l2h_begin sing_l2h 215 -->
\end{rawhtml}
$$
\hbox{\tt modulo}(A,B)=\hbox{ker}(R^r
\buildrel{A}\over{\longrightarrow}R^m/\hbox{Im}(B)) \; .
$$
\begin{rawhtml}
<!-- l2h_end sing_l2h 215 -->

\end{rawhtml}
\begin{rawhtml}

<!-- l2h_begin sing_l2h 216 -->
\end{rawhtml}
$g$, $f_1$, \dots, $f_r\in K[x_1,\ldots,x_n]$.
\begin{rawhtml}
<!-- l2h_end sing_l2h 216 -->

\end{rawhtml}
\begin{rawhtml}

<!-- l2h_begin sing_l2h 217 -->
\end{rawhtml}
$f_1$, \dots, $f_r$
\begin{rawhtml}
<!-- l2h_end sing_l2h 217 -->

\end{rawhtml}
\begin{rawhtml}

<!-- l2h_begin sing_l2h 218 -->
\end{rawhtml}
$I=\langle Y_1-f_1,\ldots,Y_r-f_r \rangle \subseteq
K[x_1,\ldots,x_n,Y_1,\ldots,Y_r]$.
\begin{rawhtml}
<!-- l2h_end sing_l2h 218 -->

\end{rawhtml}
\begin{rawhtml}

<!-- l2h_begin sing_l2h 219 -->
\end{rawhtml}
$I \cap K[Y_1,\ldots,Y_r]$
\begin{rawhtml}
<!-- l2h_end sing_l2h 219 -->

\end{rawhtml}
\begin{rawhtml}

<!-- l2h_begin sing_l2h 220 -->
\end{rawhtml}
$f_1$, \dots, $f_r$.
\begin{rawhtml}
<!-- l2h_end sing_l2h 220 -->

\end{rawhtml}
\begin{rawhtml}

<!-- l2h_begin sing_l2h 221 -->
\end{rawhtml}
$g \in K [f_1,\ldots,f_r]$.
\begin{rawhtml}
<!-- l2h_end sing_l2h 221 -->

\end{rawhtml}
\begin{rawhtml}

<!-- l2h_begin sing_l2h 222 -->
\end{rawhtml}
$g \in K[f_1,\ldots,f_r]$
\begin{rawhtml}
<!-- l2h_end sing_l2h 222 -->

\end{rawhtml}
\begin{rawhtml}

<!-- l2h_begin sing_l2h 223 -->
\end{rawhtml}
$g$
\begin{rawhtml}
<!-- l2h_end sing_l2h 223 -->

\end{rawhtml}
\begin{rawhtml}

<!-- l2h_begin sing_l2h 224 -->
\end{rawhtml}
$I$
\begin{rawhtml}
<!-- l2h_end sing_l2h 224 -->

\end{rawhtml}
\begin{rawhtml}

<!-- l2h_begin sing_l2h 225 -->
\end{rawhtml}
$X=(x_1,\ldots,x_n)$ and $Y=(Y_1,\ldots,Y_r)$ with $X>Y$
\begin{rawhtml}
<!-- l2h_end sing_l2h 225 -->

\end{rawhtml}
\begin{rawhtml}

<!-- l2h_begin sing_l2h 226 -->
\end{rawhtml}
$K[Y]$
\begin{rawhtml}
<!-- l2h_end sing_l2h 226 -->

\end{rawhtml}
\begin{rawhtml}

<!-- l2h_begin sing_l2h 227 -->
\end{rawhtml}
$[f_1,...,f_n]$ or $f_1*gen(1)+...+f_n*gen(n)$, where $gen(i)$
\begin{rawhtml}
<!-- l2h_end sing_l2h 227 -->

\end{rawhtml}
\begin{rawhtml}

<!-- l2h_begin sing_l2h 228 -->
\end{rawhtml}
$gen(i)$
\begin{rawhtml}
<!-- l2h_end sing_l2h 228 -->

\end{rawhtml}
\begin{rawhtml}

<!-- l2h_begin sing_l2h 229 -->
\end{rawhtml}
$[f_1,...,f_n]$
\begin{rawhtml}
<!-- l2h_end sing_l2h 229 -->

\end{rawhtml}
\begin{rawhtml}

<!-- l2h_begin sing_l2h 230 -->
\end{rawhtml}
$v=[f_1,...,f_n]$
\begin{rawhtml}
<!-- l2h_end sing_l2h 230 -->

\end{rawhtml}
\begin{rawhtml}

<!-- l2h_begin sing_l2h 231 -->
\end{rawhtml}
nrows($v$)
\begin{rawhtml}
<!-- l2h_end sing_l2h 231 -->

\end{rawhtml}
\begin{rawhtml}

<!-- l2h_begin sing_l2h 232 -->
\end{rawhtml}
$r$
\begin{rawhtml}
<!-- l2h_end sing_l2h 232 -->

\end{rawhtml}
\begin{rawhtml}

<!-- l2h_begin sing_l2h 233 -->
\end{rawhtml}
$f_r \not= 0$.
\begin{rawhtml}
<!-- l2h_end sing_l2h 233 -->

\end{rawhtml}
\begin{rawhtml}

<!-- l2h_begin sing_l2h 234 -->
\end{rawhtml}
$v$
\begin{rawhtml}
<!-- l2h_end sing_l2h 234 -->

\end{rawhtml}
\begin{rawhtml}

<!-- l2h_begin sing_l2h 235 -->
\end{rawhtml}
$f_i*gen(i)$
\begin{rawhtml}
<!-- l2h_end sing_l2h 235 -->

\end{rawhtml}
\begin{rawhtml}

<!-- l2h_begin sing_l2h 236 -->
\end{rawhtml}
$f_i=0$
\begin{rawhtml}
<!-- l2h_end sing_l2h 236 -->

\end{rawhtml}
\begin{rawhtml}

<!-- l2h_begin sing_l2h 237 -->
\end{rawhtml}
$v_1,...,v_k$
\begin{rawhtml}
<!-- l2h_end sing_l2h 237 -->

\end{rawhtml}
\begin{rawhtml}

<!-- l2h_begin sing_l2h 238 -->
\end{rawhtml}
nrows($M$)
\begin{rawhtml}
<!-- l2h_end sing_l2h 238 -->

\end{rawhtml}
\begin{rawhtml}

<!-- l2h_begin sing_l2h 239 -->
\end{rawhtml}
nrows($v_i$).
\begin{rawhtml}
<!-- l2h_end sing_l2h 239 -->

\end{rawhtml}
\begin{rawhtml}

<!-- l2h_begin sing_l2h 240 -->
\end{rawhtml}
$R$.
\begin{rawhtml}
<!-- l2h_end sing_l2h 240 -->

\end{rawhtml}
\begin{rawhtml}

<!-- l2h_begin sing_l2h 241 -->
\end{rawhtml}
$N$,
\begin{rawhtml}
<!-- l2h_end sing_l2h 241 -->

\end{rawhtml}
\begin{rawhtml}

<!-- l2h_begin sing_l2h 242 -->
\end{rawhtml}
$N = R^n/M$ where $n$
\begin{rawhtml}
<!-- l2h_end sing_l2h 242 -->

\end{rawhtml}
\begin{rawhtml}

<!-- l2h_begin sing_l2h 243 -->
\end{rawhtml}
$N$ and $M \subseteq R^n$
\begin{rawhtml}
<!-- l2h_end sing_l2h 243 -->

\end{rawhtml}
\begin{rawhtml}

<!-- l2h_begin sing_l2h 244 -->
\end{rawhtml}
$R^n$
\begin{rawhtml}
<!-- l2h_end sing_l2h 244 -->

\end{rawhtml}
\begin{rawhtml}

<!-- l2h_begin sing_l2h 245 -->
\end{rawhtml}
$N = R^n/M$.
\begin{rawhtml}
<!-- l2h_end sing_l2h 245 -->

\end{rawhtml}
\begin{rawhtml}

<!-- l2h_begin sing_l2h 246 -->
\end{rawhtml}
$M$,
\begin{rawhtml}
<!-- l2h_end sing_l2h 246 -->

\end{rawhtml}
\begin{rawhtml}

<!-- l2h_begin sing_l2h 247 -->
\end{rawhtml}
$N = R^n/M$ instead of $M$.
\begin{rawhtml}
<!-- l2h_end sing_l2h 247 -->

\end{rawhtml}
\begin{rawhtml}

<!-- l2h_begin sing_l2h 248 -->
\end{rawhtml}
$N$ as $N = R^n/$std($M$)).
\begin{rawhtml}
<!-- l2h_end sing_l2h 248 -->

\end{rawhtml}
\begin{rawhtml}

<!-- l2h_begin sing_l2h 249 -->
\end{rawhtml}
dim$(R^n/M)$, resp.@: dim$_k(R^n/M)$
\begin{rawhtml}
<!-- l2h_end sing_l2h 249 -->

\end{rawhtml}
\begin{rawhtml}

<!-- l2h_begin sing_l2h 250 -->
\end{rawhtml}
$N$.
\begin{rawhtml}
<!-- l2h_end sing_l2h 250 -->

\end{rawhtml}
\begin{rawhtml}

<!-- l2h_begin sing_l2h 251 -->
\end{rawhtml}
$N = R^n/M$
\begin{rawhtml}
<!-- l2h_end sing_l2h 251 -->

\end{rawhtml}
\begin{rawhtml}

<!-- l2h_begin sing_l2h 252 -->
\end{rawhtml}
$\hbox{Loc}_{(x)}K[x_1,\ldots,x_n]$.
\begin{rawhtml}
<!-- l2h_end sing_l2h 252 -->

\end{rawhtml}
\begin{rawhtml}

<!-- l2h_begin sing_l2h 253 -->
\end{rawhtml}
$\hbox{Wp}(w_1, \ldots, w_n)$
\begin{rawhtml}
<!-- l2h_end sing_l2h 253 -->

\end{rawhtml}
\begin{rawhtml}

<!-- l2h_begin sing_l2h 254 -->
\end{rawhtml}
$\hbox{Ws}(w_1, \ldots, w_n)$)
\begin{rawhtml}
<!-- l2h_end sing_l2h 254 -->

\end{rawhtml}
\begin{rawhtml}

<!-- l2h_begin sing_l2h 255 -->
\end{rawhtml}
$w_1$, $\ldots$, $w_n$ of $x_1$, $\ldots$, $x_n$.
\begin{rawhtml}
<!-- l2h_end sing_l2h 255 -->

\end{rawhtml}
\begin{rawhtml}

<!-- l2h_begin sing_l2h 256 -->
\end{rawhtml}
A monomial ordering (term ordering) on $K[x_1, \ldots, x_n]$ is
a total ordering $<$ on the
set of monomials (power products) $\{x^\alpha \mid \alpha \in \bf{N}^n\}$
which is compatible with the
natural semigroup structure, i.e., $x^\alpha < x^\beta$ implies $x^\gamma
x^\alpha < x^\gamma x^\beta$ for any $\gamma \in \bf{N}^n$.
We do not require
$<$ to be  a well ordering.
\begin{rawhtml}
<!-- l2h_end sing_l2h 256 -->

\end{rawhtml}
\begin{rawhtml}

<!-- l2h_begin sing_l2h 257 -->
\end{rawhtml}
$M$ in $GL(n,R)$,
\begin{rawhtml}
<!-- l2h_end sing_l2h 257 -->

\end{rawhtml}
\begin{rawhtml}

<!-- l2h_begin sing_l2h 258 -->
\end{rawhtml}
Global orderings are well orderings (i.e.,  \hbox{$1 < x_i$} for each variable
$x_i$), local orderings satisfy $1 > x_i$ for each variable.   If some variables are ordered globally and others locally we
call it a mixed ordering.   Local or mixed orderings are not well orderings.

Let $K$ be the ground field, \hbox{$x = (x_1, \ldots, x_n)$} the
variables and $<$ a monomial ordering, then Loc $K[x]$ denotes the
localization of $K[x]$ with respect to the multiplicatively closed set $$\{1 +
g \mid g = 0 \hbox{ or } g \in K[x]\backslash \{0\} \hbox{ and }L(g) <
1\}.$$   Here, $L(g)$ 
denotes the leading monomial of $g$, i.e., the biggest monomial of $g$ with
respect to $<$.   The result of any computation which uses standard basis
computations has to be interpreted in Loc $K[x]$.
\begin{rawhtml}
<!-- l2h_end sing_l2h 258 -->

\end{rawhtml}
\begin{rawhtml}

<!-- l2h_begin sing_l2h 259 -->
\end{rawhtml}
For all these orderings: Loc $K[x]$ = $K[x]$
\begin{rawhtml}
<!-- l2h_end sing_l2h 259 -->

\end{rawhtml}
\begin{rawhtml}

<!-- l2h_begin sing_l2h 260 -->
\end{rawhtml}
$x^\alpha < x^\beta  \Leftrightarrow  \exists\; 1 \le i \le n :
\alpha_1 = \beta_1, \ldots, \alpha_{i-1} = \beta_{i-1}, \alpha_i <
\beta_i$.
\begin{rawhtml}
<!-- l2h_end sing_l2h 260 -->

\end{rawhtml}
\begin{rawhtml}

<!-- l2h_begin sing_l2h 261 -->
\end{rawhtml}
$x^\alpha < x^\beta  \Leftrightarrow  \exists\; 1 \le i \le n :
\alpha_n = \beta_n,
    \ldots, \alpha_{i+1} = \beta_{i+1}, \alpha_i > \beta_i.$
\begin{rawhtml}
<!-- l2h_end sing_l2h 261 -->

\end{rawhtml}
\begin{rawhtml}

<!-- l2h_begin sing_l2h 262 -->
\end{rawhtml}
let $\deg(x^\alpha) = \alpha_1 + \cdots + \alpha_n,$ then
\begin{rawhtml}
<!-- l2h_end sing_l2h 262 -->

\end{rawhtml}
\begin{rawhtml}

<!-- l2h_begin sing_l2h 263 -->
\end{rawhtml}
    $x^\alpha < x^\beta \Leftrightarrow \deg(x^\alpha) < \deg(x^\beta)$ or
\begin{rawhtml}
<!-- l2h_end sing_l2h 263 -->

\end{rawhtml}
\begin{rawhtml}

<!-- l2h_begin sing_l2h 264 -->
\end{rawhtml}
    \phantom{$x^\alpha < x^\beta \Leftrightarrow $}$ \deg(x^\alpha) =
    \deg(x^\beta)$ and $\exists\ 1 \le i \le n: \alpha_n = \beta_n,
    \ldots, \alpha_{i+1} = \beta_{i+1}, \alpha_i > \beta_i.$
\begin{rawhtml}
<!-- l2h_end sing_l2h 264 -->

\end{rawhtml}
\begin{rawhtml}

<!-- l2h_begin sing_l2h 265 -->
\end{rawhtml}
    \phantom{ $x^\alpha < x^\beta \Leftrightarrow $} $\deg(x^\alpha) =
    \deg(x^\beta)$ and $\exists\ 1 \le i \le n:\alpha_1 = \beta_1,
    \ldots, \alpha_{i-1} = \beta_{i-1}, \alpha_i < \beta_i.$
\begin{rawhtml}
<!-- l2h_end sing_l2h 265 -->

\end{rawhtml}
\begin{rawhtml}

<!-- l2h_begin sing_l2h 266 -->
\end{rawhtml}
let $w_1, \ldots, w_n$ be positive integers. Then ${\tt wp}(w_1, \ldots,
w_n)$
\begin{rawhtml}
<!-- l2h_end sing_l2h 266 -->

\end{rawhtml}
\begin{rawhtml}

<!-- l2h_begin sing_l2h 267 -->
\end{rawhtml}
$\deg(x^\alpha) = w_1 \alpha_1 + \cdots + w_n\alpha_n.$
\begin{rawhtml}
<!-- l2h_end sing_l2h 267 -->

\end{rawhtml}
\begin{rawhtml}

<!-- l2h_begin sing_l2h 268 -->
\end{rawhtml}
let $w_1, \ldots, w_n$ be positive integers. Then ${\tt Wp}(w_1, \ldots,
w_n)$
\begin{rawhtml}
<!-- l2h_end sing_l2h 268 -->

\end{rawhtml}
\begin{rawhtml}

<!-- l2h_begin sing_l2h 269 -->
\end{rawhtml}
Loc $K[x]$ = $K[x]_{(x)}$,
\begin{rawhtml}
<!-- l2h_end sing_l2h 269 -->

\end{rawhtml}
\begin{rawhtml}

<!-- l2h_begin sing_l2h 270 -->
\end{rawhtml}
$K[x]$
\begin{rawhtml}
<!-- l2h_end sing_l2h 270 -->

\end{rawhtml}
\begin{rawhtml}

<!-- l2h_begin sing_l2h 271 -->
\end{rawhtml}
\ $(x_1, ..., x_n)$.
\begin{rawhtml}
<!-- l2h_end sing_l2h 271 -->

\end{rawhtml}
\begin{rawhtml}

<!-- l2h_begin sing_l2h 272 -->
\end{rawhtml}
$x^\alpha < x^\beta  \Leftrightarrow  \exists\; 1 \le i \le n :
\alpha_1 = \beta_1, \ldots, \alpha_{i-1} = \beta_{i-1}, \alpha_i >
\beta_i$.
\begin{rawhtml}
<!-- l2h_end sing_l2h 272 -->

\end{rawhtml}
\begin{rawhtml}

<!-- l2h_begin sing_l2h 273 -->
\end{rawhtml}
    $x^\alpha < x^\beta \Leftrightarrow \deg(x^\alpha) > \deg(x^\beta)$ or
\begin{rawhtml}
<!-- l2h_end sing_l2h 273 -->

\end{rawhtml}
\begin{rawhtml}

<!-- l2h_begin sing_l2h 274 -->
\end{rawhtml}
    \phantom{ $x^\alpha < x^\beta \Leftrightarrow$}$ \deg(x^\alpha) =
    \deg(x^\beta)$ and $\exists\ 1 \le i \le n: \alpha_n = \beta_n,
    \ldots, \alpha_{i+1} = \beta_{i+1}, \alpha_i > \beta_i.$
\begin{rawhtml}
<!-- l2h_end sing_l2h 274 -->

\end{rawhtml}
\begin{rawhtml}

<!-- l2h_begin sing_l2h 275 -->
\end{rawhtml}
    \phantom{ $ x^\alpha < x^\beta \Leftrightarrow$}$ \deg(x^\alpha) =
    \deg(x^\beta)$ and $\exists\ 1 \le i \le n:\alpha_1 = \beta_1,
    \ldots, \alpha_{i-1} = \beta_{i-1}, \alpha_i < \beta_i.$
\begin{rawhtml}
<!-- l2h_end sing_l2h 275 -->

\end{rawhtml}
\begin{rawhtml}

<!-- l2h_begin sing_l2h 276 -->
\end{rawhtml}
${\tt ws}(w_1, \ldots, w_n),\; w_1$
\begin{rawhtml}
<!-- l2h_end sing_l2h 276 -->

\end{rawhtml}
\begin{rawhtml}

<!-- l2h_begin sing_l2h 277 -->
\end{rawhtml}
$w_2,\ldots,w_n$
\begin{rawhtml}
<!-- l2h_end sing_l2h 277 -->

\end{rawhtml}
\begin{rawhtml}

<!-- l2h_begin sing_l2h 278 -->
\end{rawhtml}
${\tt Ws}(w_1, \ldots, w_n),\; w_1$
\begin{rawhtml}
<!-- l2h_end sing_l2h 278 -->

\end{rawhtml}
\begin{rawhtml}

<!-- l2h_begin sing_l2h 279 -->
\end{rawhtml}
$\{ x^a e_i  \mid  a \in N^n, 1 \leq i \leq r \}$ in Loc $K[x]^r$ = Loc
$K[x]e_1 
+ \ldots +$Loc $K[x]e_r$, where $e_1, \ldots, e_r$ denote the canonical
generators of Loc $K[x]^r$, the r-fold direct sum of Loc $K[x]$.
(The function {\tt gen(i)} yields $e_i$).
\begin{rawhtml}
<!-- l2h_end sing_l2h 279 -->

\end{rawhtml}
\begin{rawhtml}

<!-- l2h_begin sing_l2h 280 -->
\end{rawhtml}
 Loc $K[x]^r$
\begin{rawhtml}
<!-- l2h_end sing_l2h 280 -->

\end{rawhtml}
\begin{rawhtml}

<!-- l2h_begin sing_l2h 281 -->
\end{rawhtml}
Loc $K[x]$
\begin{rawhtml}
<!-- l2h_end sing_l2h 281 -->

\end{rawhtml}
\begin{rawhtml}

<!-- l2h_begin sing_l2h 282 -->
\end{rawhtml}
$<_m = (<,C)$ denotes the module ordering (giving priority to the coefficients):
\begin{rawhtml}
<!-- l2h_end sing_l2h 282 -->

\end{rawhtml}
\begin{rawhtml}

<!-- l2h_begin sing_l2h 283 -->
\end{rawhtml}
\quad  \quad  $x^\alpha e_i <_m x^\beta e_j \Leftrightarrow x^\alpha <
x^\beta$ or ($x^\alpha = x^\beta $ and $ i < j$).
\begin{rawhtml}
<!-- l2h_end sing_l2h 283 -->

\end{rawhtml}
\begin{rawhtml}

<!-- l2h_begin sing_l2h 284 -->
\end{rawhtml}
$<_m = (C, <)$ denotes the module ordering (giving priority to the component):
\begin{rawhtml}
<!-- l2h_end sing_l2h 284 -->

\end{rawhtml}
\begin{rawhtml}

<!-- l2h_begin sing_l2h 285 -->
\end{rawhtml}
\quad \quad   $x^\alpha e_i <_m x^\beta e_j \Leftrightarrow i < j$ or ($
i = j $ and $ x^\alpha < x^\beta $).
\begin{rawhtml}
<!-- l2h_end sing_l2h 285 -->

\end{rawhtml}
\begin{rawhtml}

<!-- l2h_begin sing_l2h 286 -->
\end{rawhtml}
$<_m = (<,c)$ denotes the module ordering (giving priority to the coefficients):
\begin{rawhtml}
<!-- l2h_end sing_l2h 286 -->

\end{rawhtml}
\begin{rawhtml}

<!-- l2h_begin sing_l2h 287 -->
\end{rawhtml}
\quad \quad $x^\alpha e_i <_m x^\beta e_j \Leftrightarrow x^\alpha <
x^\beta$ or ($x^\alpha = x^\beta $ and $ i > j$).
\begin{rawhtml}
<!-- l2h_end sing_l2h 287 -->

\end{rawhtml}
\begin{rawhtml}

<!-- l2h_begin sing_l2h 288 -->
\end{rawhtml}
$<_m = (c, <)$ denotes the module ordering (giving priority to the component):
\begin{rawhtml}
<!-- l2h_end sing_l2h 288 -->

\end{rawhtml}
\begin{rawhtml}

<!-- l2h_begin sing_l2h 289 -->
\end{rawhtml}
\quad \quad   $x^\alpha e_i <_m x^\beta e_j \Leftrightarrow i > j$ or ($
i = j $ and $ x^\alpha < x^\beta $).
\begin{rawhtml}
<!-- l2h_end sing_l2h 289 -->

\end{rawhtml}
\begin{rawhtml}

<!-- l2h_begin sing_l2h 290 -->
\end{rawhtml}
The output of a vector $v$ in $K[x]^r$ with components $v_1,
\ldots, v_r$ has the format $v_1 * gen(1) + \ldots + v_r * gen(r)$
\begin{rawhtml}
<!-- l2h_end sing_l2h 290 -->

\end{rawhtml}
\begin{rawhtml}

<!-- l2h_begin sing_l2h 291 -->
\end{rawhtml}
In this case a vector is written as $[v_1, \ldots, v_r]$.
\begin{rawhtml}
<!-- l2h_end sing_l2h 291 -->

\end{rawhtml}
\begin{rawhtml}

<!-- l2h_begin sing_l2h 292 -->
\end{rawhtml}
$(n \times n)$-matrix
\begin{rawhtml}
<!-- l2h_end sing_l2h 292 -->

\end{rawhtml}
\begin{rawhtml}

<!-- l2h_begin sing_l2h 293 -->
\end{rawhtml}
$M_1, \ldots, M_n$ the rows of $M$.
\begin{rawhtml}
<!-- l2h_end sing_l2h 293 -->

\end{rawhtml}
\begin{rawhtml}

<!-- l2h_begin sing_l2h 294 -->
\end{rawhtml}
\quad \quad $x^a < x^b \Leftrightarrow \exists\  1 \leq i \leq n :
M_1 a = \; M_1 b, \ldots, M_{i-1} a = \; M_{i-1} b$ and $M_i a < \; M_i b$.
\begin{rawhtml}
<!-- l2h_end sing_l2h 294 -->

\end{rawhtml}
\begin{rawhtml}

<!-- l2h_begin sing_l2h 295 -->
\end{rawhtml}
$x^a < x^b$
if and only if $M a$ is smaller than $M b$
\begin{rawhtml}
<!-- l2h_end sing_l2h 295 -->

\end{rawhtml}
\begin{rawhtml}

<!-- l2h_begin sing_l2h 296 -->
\end{rawhtml}

$\quad$ lp:
$\left(\matrix{
 1 & 0 & 0 \cr
 0 & 1 & 0 \cr
 0 & 0 & 1 \cr
 }\right)$
\quad dp:
$\left(\matrix{
 1 & 1 & 1 \cr
 0 & 0 &-1 \cr
 0 &-1 & 0 \cr
 }\right)$
\quad Dp:
$\left(\matrix{
 1 & 1 & 1 \cr
 1 & 0 & 0 \cr
 0 & 1 & 0 \cr
 }\right)$

$\quad$ wp(1,2,3):
$\left(\matrix{
 1 & 2 & 3 \cr
 0 & 0 &-1 \cr
 0 &-1 & 0 \cr
 }\right)$
\quad Wp(1,2,3):
$\left(\matrix{
 1 & 2 & 3 \cr
 1 & 0 & 0 \cr
 0 & 1 & 0 \cr
 }\right)$

$\quad$ ls:
$\left(\matrix{
-1 & 0 & 0 \cr
 0 &-1 & 0 \cr
 0 & 0 &-1 \cr
 }\right)$
\quad ds:
$\left(\matrix{
-1 &-1 &-1 \cr
 0 & 0 &-1 \cr
 0 &-1 & 0 \cr
 }\right)$
\quad Ds:
$\left(\matrix{
-1 &-1 &-1 \cr
 1 & 0 & 0 \cr
 0 & 1 & 0 \cr
 }\right)$

$\quad$ ws(1,2,3):
$\left(\matrix{
-1 &-2 &-3 \cr
 0 & 0 &-1 \cr
 0 &-1 & 0 \cr
 }\right)$
\quad Ws(1,2,3):
$\left(\matrix{
-1 &-2 &-3 \cr
 1 & 0 & 0 \cr
 0 & 1 & 0 \cr
 }\right)$
\begin{rawhtml}
<!-- l2h_end sing_l2h 296 -->

\end{rawhtml}
\begin{rawhtml}

<!-- l2h_begin sing_l2h 297 -->
\end{rawhtml}
$\quad$ (dp(3), wp(1,2,3)):
$\left(\matrix{
1&  1&  1&  0&  0&  0 \cr
0&  0&  -1&  0&  0&  0 \cr
0&  -1&  0&  0&  0&  0 \cr
0&  0&  0&  1&  2&  3 \cr
0&  0&  0&  0&  0&  -1 \cr
0&  0&  0&  0&  -1&  0 \cr
 }\right)$

$\quad$ (Dp(3), ds(3)):
$\left(\matrix{
1&  1&  1&  0&  0&  0 \cr
1&  0&  0&  0&  0&  0 \cr
0&  1&  0&  0&  0&  0 \cr
0&  0&  0&  -1&  -1&  -1 \cr
0&  0&  0&  0&  0&  -1 \cr
0&  0&  0&  0&  -1&  0 \cr
 }\right)$
\begin{rawhtml}
<!-- l2h_end sing_l2h 297 -->

\end{rawhtml}
\begin{rawhtml}

<!-- l2h_begin sing_l2h 298 -->
\end{rawhtml}
$\quad$ (dp(3), a(1,2,3),dp(3)):
$\left(\matrix{
1&  1&  1&  0&  0&  0 \cr
0&  0&  -1&  0&  0&  0 \cr
0&  -1&  0&  0&  0&  0 \cr
0&  0&  0&  1&  2&  3 \cr
0&  0&  0&  1&  1&  1 \cr
0&  0&  0&  0&  0&  -1 \cr
0&  0&  0&  0&  -1&  0 \cr
 }\right)$

$\quad$ (a(1,2,3,4,5),Dp(3), ds(3)):
$\left(\matrix{
1&  2&  3&  4&  5&  0 \cr
1&  1&  1&  0&  0&  0 \cr
1&  0&  0&  0&  0&  0 \cr
0&  1&  0&  0&  0&  0 \cr
0&  0&  0&  -1&  -1&  -1 \cr
0&  0&  0&  0&  0 & -1 \cr
0&  0&  0&  0&  -1&  0 \cr
 }\right)$
\begin{rawhtml}
<!-- l2h_end sing_l2h 298 -->

\end{rawhtml}
\begin{rawhtml}

<!-- l2h_begin sing_l2h 299 -->
\end{rawhtml}
$n \times n$
\begin{rawhtml}
<!-- l2h_end sing_l2h 299 -->

\end{rawhtml}
\begin{rawhtml}

<!-- l2h_begin sing_l2h 300 -->
\end{rawhtml}
$x = (x_1, \ldots, x_n)$ and $y = (y_1, \ldots, y_m)$
\begin{rawhtml}
<!-- l2h_end sing_l2h 300 -->

\end{rawhtml}
\begin{rawhtml}

<!-- l2h_begin sing_l2h 301 -->
\end{rawhtml}
$<_1$ a monomial
ordering on $K[x]$ and $<_2$ a monomial ordering on $K[y]$.   The product
ordering (or block ordering) $<\ := (<_1,<_2)$ on $K[x,y]$ is the following:
\begin{rawhtml}
<!-- l2h_end sing_l2h 301 -->

\end{rawhtml}
\begin{rawhtml}

<!-- l2h_begin sing_l2h 302 -->
\end{rawhtml}
\quad \quad $x^a y^b < x^A y^B \Leftrightarrow x^a <_1 x^A $ or ($x^a =
x^A$ and $y^b <_2 y^B$).
\begin{rawhtml}
<!-- l2h_end sing_l2h 302 -->

\end{rawhtml}
\begin{rawhtml}

<!-- l2h_begin sing_l2h 303 -->
\end{rawhtml}
${\tt a}(w_1, \ldots, w_n),\; $
\begin{rawhtml}
<!-- l2h_end sing_l2h 303 -->

\end{rawhtml}
\begin{rawhtml}

<!-- l2h_begin sing_l2h 304 -->
\end{rawhtml}
$w_1,\ldots,w_n$
\begin{rawhtml}
<!-- l2h_end sing_l2h 304 -->

\end{rawhtml}
\begin{rawhtml}

<!-- l2h_begin sing_l2h 305 -->
\end{rawhtml}
$\deg(x^\alpha) = w_1 \alpha_1 + \cdots + w_n\alpha_n$
\begin{rawhtml}
<!-- l2h_end sing_l2h 305 -->

\end{rawhtml}
\begin{rawhtml}

<!-- l2h_begin sing_l2h 306 -->
\end{rawhtml}
    $$\deg(x^\alpha) < \deg(x^\beta) \Rightarrow x^\alpha < x^\beta,$$
\begin{rawhtml}
<!-- l2h_end sing_l2h 306 -->

\end{rawhtml}
\begin{rawhtml}

<!-- l2h_begin sing_l2h 307 -->
\end{rawhtml}
    $$\deg(x^\alpha) > \deg(x^\beta) \Rightarrow x^\alpha > x^\beta. $$
\begin{rawhtml}
<!-- l2h_end sing_l2h 307 -->

\end{rawhtml}
\begin{rawhtml}

<!-- l2h_begin sing_l2h 308 -->
\end{rawhtml}
Let $R = \hbox{Loc}_< K[\underline{x}]$ and let $I$ be a submodule of $R^r$.
Note that for r=1 this means that $I$ is an ideal in $R$.
Denote by $L(I)$ the submodule of $R^r$ generated by the leading terms 
of elements of $I$, i.e. by $\left\{L(f) \mid f \in I\right\}$.
Then $f_1, \ldots, f_s \in I$ is called a {\bf standard basis} of $I$ 
if $L(f_1), \ldots, L(f_s)$ generate $L(I)$.
\begin{rawhtml}
<!-- l2h_end sing_l2h 308 -->

\end{rawhtml}
\begin{rawhtml}

<!-- l2h_begin sing_l2h 309 -->
\end{rawhtml}
A function $\hbox{NF} : R^r \times \{G \mid G\ \hbox{ a standard
basis}\} \to R^r, (p,G) \mapsto \hbox{NF}(p|G)$, is called a {\bf normal
form} if for any $p \in R^r$ and any standard basis $G$ the following
holds: if $\hbox{NF}(p|G) \not= 0$ then $L(g)$ does not divide
$L(\hbox{NF}(p|G))$ for all $g \in G$.

\noindent
$\hbox{NF}(p|G)$ is called a {\bf normal form of} $p$ {\bf with
respect to} $G$ (note that such a function is not unique).
\begin{rawhtml}
<!-- l2h_end sing_l2h 309 -->

\end{rawhtml}
\begin{rawhtml}

<!-- l2h_begin sing_l2h 310 -->
\end{rawhtml}
For a standard basis $G$ of $I$ the following holds: 
$f \in I$ if and only if $\hbox{NF}(f,G) = 0$.
\begin{rawhtml}
<!-- l2h_end sing_l2h 310 -->

\end{rawhtml}
\begin{rawhtml}

<!-- l2h_begin sing_l2h 311 -->
\end{rawhtml}
Let \hbox{$I \subseteq K[\underline{x}]^r$} be a homogeneous module, then the Hilbert function
$H_I$ of $I$ (see below)
and the Hilbert function $H_{L(I)}$ of the leading module $L(I)$
coincide, i.e.,
$H_I=H_{L(I)}$.
\begin{rawhtml}
<!-- l2h_end sing_l2h 311 -->

\end{rawhtml}
\begin{rawhtml}

<!-- l2h_begin sing_l2h 312 -->
\end{rawhtml}
Let M $=\bigoplus_i M_i$ be a graded module over $K[x_1,..,x_n]$ with 
respect to weights $(w_1,..w_n)$.
The {\bf Hilbert function} of $M$, $H_M$, is defined (on the integers) by
$$H_M(k) :=dim_K M_k.$$
The {\bf Hilbert-Poincare series}  of $M$ is the power series
$$\hbox{HP}_M(t) :=\sum_{i=-\infty}^\infty
H_M(i)t^i=\sum_{i=-\infty}^\infty dim_K M_i \cdot t^i.$$
It turns out that $\hbox{HP}_M(t)$ can be written in two useful ways
for weights $(1,..,1)$:
$$\hbox{HP}_M(t)={Q(t)\over (1-t)^n}={P(t)\over (1-t)^{dim(M)}}$$
where $Q(t)$ and $P(t)$ are polynomials in ${\bf Z}[t]$.
$Q(t)$ is called the {\bf first Hilbert series},
and $P(t)$ the {\bf second Hilbert series}.
If \hbox{$P(t)=\sum_{k=0}^N a_k t^k$}, and \hbox{$d = dim(M)$},
then \hbox{$H_M(s)=\sum_{k=0}^N a_k$ ${d+s-k-1}\choose{d-1}$}
(the {\bf Hilbert polynomial}) for $s \ge N$.
\begin{rawhtml}
<!-- l2h_end sing_l2h 312 -->

\end{rawhtml}
\begin{rawhtml}

<!-- l2h_begin sing_l2h 313 -->
\end{rawhtml}
Generalizing these to quasihomogeneous modules we get
$$\hbox{HP}_M(t)={Q(t)\over {\Pi_{i=1}^n(1-t^{w_i})}}$$
where $Q(t)$ is a polynomial in ${\bf Z}[t]$.
$Q(t)$ is called the {\bf first (weighted) Hilbert series} of M.
\begin{rawhtml}
<!-- l2h_end sing_l2h 313 -->

\end{rawhtml}
\begin{rawhtml}

<!-- l2h_begin sing_l2h 314 -->
\end{rawhtml}
Let $R$ be a quotient of $\hbox{Loc}_< K[\underline{x}]$ and let \hbox{$I=(g_1, ..., g_s)$} be a submodule of $R^r$.
Then the {\bf module of syzygies} (or {\bf 1st syzygy module}, {\bf module of relations}) of $I$, syz($I$), is defined to be the kernel of the map \hbox{$R^s \rightarrow R^r,\; \sum_{i=1}^s w_ie_i \mapsto \sum_{i=1}^s w_ig_i$.}
\begin{rawhtml}
<!-- l2h_end sing_l2h 314 -->

\end{rawhtml}
\begin{rawhtml}

<!-- l2h_begin sing_l2h 315 -->
\end{rawhtml}
$(k-1)$-st
\begin{rawhtml}
<!-- l2h_end sing_l2h 315 -->

\end{rawhtml}
\begin{rawhtml}

<!-- l2h_begin sing_l2h 316 -->
\end{rawhtml}
Note, that the syzygy modules of $I$ depend on a choice of generators $g_1, ..., g_s$.
But one can show that they depend on $I$ uniquely up to direct summands.
\begin{rawhtml}
<!-- l2h_end sing_l2h 316 -->

\end{rawhtml}
\begin{rawhtml}

<!-- l2h_begin sing_l2h 317 -->
\end{rawhtml}
Let $I=(g_1,...,g_s)\subseteq R^r$ and $M= R^r/I$.
A {\bf free resolution of $M$} is a long exact sequence
$$...\longrightarrow F_2 \buildrel{A_2}\over{\longrightarrow} F_1
\buildrel{A_1}\over{\longrightarrow} F_0\longrightarrow M\longrightarrow
0,$$
\begin{rawhtml}
<!-- l2h_end sing_l2h 317 -->

\end{rawhtml}
\begin{rawhtml}

<!-- l2h_begin sing_l2h 318 -->
\end{rawhtml}
$R=\hbox{Loc}_< K[\underline{x}]$
\begin{rawhtml}
<!-- l2h_end sing_l2h 318 -->

\end{rawhtml}
\begin{rawhtml}

<!-- l2h_begin sing_l2h 319 -->
\end{rawhtml}
Let $R$ be a graded ring (e.g., $R = \hbox{Loc}_< K[\underline{x}]$) and
let $I \subset R^r$ be a graded submodule. Let
$$
  R^r = \bigoplus_a R\cdot e_{a,0} \buildrel A_1 \over \longleftarrow
        \bigoplus_a R\cdot e_{a,1} \longleftarrow \ldots \longleftarrow
        \bigoplus_a R\cdot e_{a,n} \longleftarrow 0
$$
be a minimal free resolution of $R^n/I$ considered with homogeneous maps
of degree 0. Then the {\bf graded Betti number} $b_{i,j}$ of $R^r/I$ is
the minimal number of generators $e_{a,j}$ in degree $i+j$ of the $j$-th
syzygy module of $R^r/I$ (i.e., the $(j-1)$-st syzygy module of
$I$). Note, that by definition the $0$-th syzygy module of $R^r/I$ is $R^r$
and the 1st syzygy module of $R^r/I$ is $I$.
\begin{rawhtml}
<!-- l2h_end sing_l2h 319 -->

\end{rawhtml}
\begin{rawhtml}

<!-- l2h_begin sing_l2h 320 -->
\end{rawhtml}
$s$
\begin{rawhtml}
<!-- l2h_end sing_l2h 320 -->

\end{rawhtml}
\begin{rawhtml}

<!-- l2h_begin sing_l2h 321 -->
\end{rawhtml}
$$
    \hbox{deg}(e_{a,j}) \le s+j-1 \quad \hbox{for all $j$.}
$$
\begin{rawhtml}
<!-- l2h_end sing_l2h 321 -->

\end{rawhtml}
\begin{rawhtml}

<!-- l2h_begin sing_l2h 322 -->
\end{rawhtml}
Let $<$ be the lexicographical ordering on $R=K[x_1,...,x_n]$ with $x_1
< ... < x_n$.
For $f \in R$ let lvar($f$) (the leading variable of $f$) be the largest
variable in $f$,
i.e., if $f=a_s(x_1,...,x_{k-1})x_k^s+...+a_0(x_1,...,x_{k-1})$ for some
$k \leq n$ then lvar$(f)=x_k$.

Moreover, let
\hbox{ini}$(f):=a_s(x_1,...,x_{k-1})$. The pseudo remainder
$r=\hbox{prem}(g,f)$ of $g$ with respect to $f$ is
defined by the equality $\hbox{ini}(f)^a\cdot g = qf+r$ with
$\hbox{deg}_{lvar(f)}(r)<\hbox{deg}_{lvar(f)}(f)$ and $a$
minimal.

A set $T=\{f_1,...,f_r\} \subset R$ is called triangular if
$\hbox{lvar}(f_1)<...<\hbox{lvar}(f_r)$. Moreover, let $ U \subset T $,
then $(T,U)$ is called a triangular system, if $T$ is a triangular set
such that $\hbox{ini}(T)$ does not vanish on $V(T) \setminus V(U)
(=:V(T\setminus U))$.

$T$ is called irreducible if for every $i$ there are no
$d_i$,$f_i'$,$f_i''$ such that
$$   \hbox{lvar}(d_i)<\hbox{lvar}(f_i) =
\hbox{lvar}(f_i')=\hbox{lvar}(f_i''),$$
$$   0 \not\in \hbox{prem}(\{ d_i, \hbox{ini}(f_i'),
\hbox{ini}(f_i'')\},\{ f_1,...,f_{i-1}\}),$$
$$\hbox{prem}(d_if_i-f_i'f_i'',\{f_1,...,f_{i-1}\})=0.$$
Furthermore, $(T,U)$ is called irreducible if $T$ is irreducible.

The main result on triangular sets is the following:
let $G=\{g_1,...,g_s\} \subset R$ then there are irreducible triangular sets $T_1,...,T_l$
such that $V(G)=\bigcup_{i=1}^{l}(V(T_i\setminus I_i))$
where $I_i=\{\hbox{ini}(f) \mid f \in T_i \}$. Such a set
$\{T_1,...,T_l\}$ is called an {\bf irreducible characteristic series} of
the ideal $(G)$.
\begin{rawhtml}
<!-- l2h_end sing_l2h 322 -->

\end{rawhtml}
\begin{rawhtml}

<!-- l2h_begin sing_l2h 323 -->
\end{rawhtml}
Let $f\colon(C^{n+1},0)\rightarrow(C,0)$ be a complex isolated hypersurface singularity given by a polynomial with algebraic coefficients which we also denote by $f$.
Let $O=C[x_0,\ldots,x_n]_{(x_0,\ldots,x_n)}$ be the local ring at the origin and $J_f$ the Jacobian ideal of $f$.

A {\bf Milnor representative} of $f$ defines a differentiable fibre bundle over the punctured disc with fibres of homotopy type of $\mu$ $n$-spheres.
The $n$-th cohomology bundle is a flat vector bundle of dimension $n$ and carries a natural flat connection with covariant derivative $\partial_t$.
The {\bf monodromy operator} is the action of a positively oriented generator of the fundamental group of the puctured disc on the Milnor fibre.
Sections in the cohomology bundle of {\bf moderate growth} at $0$ form a regular $D=C\{t\}[\partial_t]$-module $G$, the {\bf Gauss-Manin connection}.

By integrating along flat multivalued families of cycles, one can consider fibrewise global holomorphic differential forms as elements of $G$.
This factors through an inclusion of the {\bf Brieskorn lattice} $H'':=\Omega^{n+1}_{C^{n+1},0}/df\wedge d\Omega^{n-1}_{C^{n+1},0}$ in $G$.

The $D$-module structure defines the {\bf V-filtration} $V$ on $G$ by $V^\alpha:=\sum_{\beta\ge\alpha}C\{t\}ker(t\partial_t-\beta)^{n+1}$.
The Brieskorn lattice defines the {\bf Hodge filtration} $F$ on $G$ by $F_k=\partial_t^kH''$ which comes from the {\bf mixed Hodge structure} on the Milnor fibre.
Note that $F_{-1}=H'$.

The induced V-filtration on the Brieskorn lattice determines the {\bf singularity spectrum} $Sp$ by $Sp(\alpha):=\dim_CGr_V^\alpha Gr^F_0G$.
The spectrum consists of $\mu$ rational numbers $\alpha_1,\dots,\alpha_\mu$ such that $e^{2\pi i\alpha_1},\dots,e^{2\pi i\alpha_\mu}$ are the eigenvalues of the monodromy.
These {\bf spectral numbers} lie in the open interval $(-1,n)$, symmetric about the midpoint $(n-1)/2$.

The spectrum is constant under $\mu$-constant deformations and has the following semicontinuity property:
The number of spectral numbers in an interval $(a,a+1]$ of all singularities of a small deformation of $f$ is greater or equal to that of f in this interval.
For semiquasihomogeneous singularities, this also holds for intervals of the form $(a,a+1)$.

Two given isolated singularities $f$ and $g$ determine two spectra and from these spectra we get an integer.
This integer is the maximal positive integer $k$ such that the semicontinuity holds for the spectrum of $f$ and $k$ times the spectrum of $g$.
These numbers give bounds for the maximal number of isolated singularities of a specific type on a hypersurface $X\subset{P}^n$ of degree $d$: 
such a hypersurface has a smooth hyperplane section, and the complement is a small deformation of a cone over this hyperplane section.
The cone itself being a $\mu$-constant deformation of $x_0^d+\dots+x_n^d=0$, the singularities are bounded by the spectrum of $x_0^d+\dots+x_n^d$.

Using the library {\tt gaussman.lib} one can compute the {\bf monodromy}, the V-filtration on $H''/H'$, and the spectrum.
\begin{rawhtml}
<!-- l2h_end sing_l2h 323 -->

\end{rawhtml}
\begin{rawhtml}

<!-- l2h_begin sing_l2h 324 -->
\end{rawhtml}
$f=x^5+x^2y^2+y^5$
\begin{rawhtml}
<!-- l2h_end sing_l2h 324 -->

\end{rawhtml}
\begin{rawhtml}

<!-- l2h_begin sing_l2h 325 -->
\end{rawhtml}
$\exp(2\pi iM)$
\begin{rawhtml}
<!-- l2h_end sing_l2h 325 -->

\end{rawhtml}
\begin{rawhtml}

<!-- l2h_begin sing_l2h 326 -->
\end{rawhtml}
$H''/H'$
\begin{rawhtml}
<!-- l2h_end sing_l2h 326 -->

\end{rawhtml}
\begin{rawhtml}

<!-- l2h_begin sing_l2h 327 -->
\end{rawhtml}
$O/J_f\cong H''/H'$
\begin{rawhtml}
<!-- l2h_end sing_l2h 327 -->

\end{rawhtml}
\begin{rawhtml}

<!-- l2h_begin sing_l2h 328 -->
\end{rawhtml}
If the principal part of $f$ is $C$-nondegenerate, one can compute the spectrum using the library {\tt spectrum.lib}.
In this case, the V-filtration on $H''$ coincides with the Newton-filtration on $H''$ which allows to compute the spectrum more efficiently.
\begin{rawhtml}
<!-- l2h_end sing_l2h 328 -->

\end{rawhtml}
\begin{rawhtml}

<!-- l2h_begin sing_l2h 329 -->
\end{rawhtml}
$\tilde{E}_6$ on a surface $X\subset{P}^3$
\begin{rawhtml}
<!-- l2h_end sing_l2h 329 -->

\end{rawhtml}
\begin{rawhtml}

<!-- l2h_begin sing_l2h 330 -->
\end{rawhtml}
$\mu(f)$, the geometric genus $p_g(f)$
\begin{rawhtml}
<!-- l2h_end sing_l2h 330 -->

\end{rawhtml}
\begin{rawhtml}

<!-- l2h_begin sing_l2h 331 -->
\end{rawhtml}
$x^7+y^7+z^7=0$
\begin{rawhtml}
<!-- l2h_end sing_l2h 331 -->

\end{rawhtml}
\begin{rawhtml}

<!-- l2h_begin sing_l2h 332 -->
\end{rawhtml}
${3 \over 7}, {4 \over 7}, {5 \over 7}, {6 \over 7}, {1 \over 1},
{8 \over 7}, {9 \over 7}, {10 \over 7}, {11 \over 7}, {12 \over 7},
{13 \over 7}, {2 \over 1}, {15 \over 7}, {16 \over 7}, {17 \over 7},
{18 \over 7}$
\begin{rawhtml}
<!-- l2h_end sing_l2h 332 -->

\end{rawhtml}
\begin{rawhtml}

<!-- l2h_begin sing_l2h 333 -->
\end{rawhtml}
The singularities of type $\tilde{E}_6$ form a
$\mu$-constant one parameter family given by
$x^3+y^3+z^3+\lambda xyz=0,\quad \lambda^3\neq-27$.
\begin{rawhtml}
<!-- l2h_end sing_l2h 333 -->

\end{rawhtml}
\begin{rawhtml}

<!-- l2h_begin sing_l2h 334 -->
\end{rawhtml}
$x^3+y^3+z^3$.
\begin{rawhtml}
<!-- l2h_end sing_l2h 334 -->

\end{rawhtml}
\begin{rawhtml}

<!-- l2h_begin sing_l2h 335 -->
\end{rawhtml}
$\tilde{E}_6$ on a septic in $P^3$. But $x^7+y^7+z^7$
\begin{rawhtml}
<!-- l2h_end sing_l2h 335 -->

\end{rawhtml}
\begin{rawhtml}

<!-- l2h_begin sing_l2h 336 -->
\end{rawhtml}
$\tilde{E}_6$.
\begin{rawhtml}
<!-- l2h_end sing_l2h 336 -->

\end{rawhtml}
\begin{rawhtml}

<!-- l2h_begin sing_l2h 337 -->
\end{rawhtml}
Let $A$ denote an $m\times n$ matrix with integral coefficients. For $u
\in Z\!\!\! Z^n$, we define $u^+,u^-$ to be the uniquely determined
vectors with nonnegative coefficients and disjoint support (i.e.,
$u_i^+=0$ or $u_i^-=0$ for each component $i$) such that
$u=u^+-u^-$. For $u\geq 0$ component-wise, let $x^u$ denote the monomial
$x_1^{u_1}\cdot\ldots\cdot x_n^{u_n}\in K[x_1,\ldots,x_n]$.

The ideal
$$ I_A:=<x^{u^+}-x^{u^-} | u\in\ker(A)\cap Z\!\!\! Z^n>\ \subset
K[x_1,\ldots,x_n] $$
is called a \bf toric ideal. \rm

The first problem in computing toric ideals is to find a finite
generating set: Let $v_1,\ldots,v_r$ be a lattice basis of $\ker(A)\cap
Z\!\!\! Z^n$ (i.e, a basis of the $Z\!\!\! Z$-module). Then
$$ I_A:=I:(x_1\cdot\ldots\cdot x_n)^\infty $$
where
$$ I=<x^{v_i^+}-x^{v_i^-}|i=1,\ldots,r> $$
\begin{rawhtml}
<!-- l2h_end sing_l2h 337 -->

\end{rawhtml}
\begin{rawhtml}

<!-- l2h_begin sing_l2h 338 -->
\end{rawhtml}
section Algorithms.
\begin{rawhtml}
<!-- l2h_end sing_l2h 338 -->

\end{rawhtml}
\begin{rawhtml}

<!-- l2h_begin sing_l2h 339 -->
\end{rawhtml}
computes $I_A$ via the
extended matrix $B=(I_m|A)$,
where $I_m$ is the $m\times m$ unity matrix. A lattice basis of $B$ is
given by the set of vectors $(a^j,-e_j)\in Z\!\!\! Z^{m+n}$, where $a^j$
is the $j$-th row of $A$ and $e_j$ the $j$-th coordinate vector. We
look at the ideal in $K[y_1,\ldots,y_m,x_1,\ldots,x_n]$ corresponding to
these vectors, namely
$$ I_1=<y^{a_j^+}- x_j y^{a_j^-} | j=1,\ldots, n>.$$
We introduce a further variable $t$ and adjoin the binomial $t\cdot
y_1\cdot\ldots\cdot y_m -1$ to the generating set of $I_1$, obtaining
an ideal $I_2$ in the polynomial ring $K[t,
y_1,\ldots,y_m,x_1,\ldots,x_n]$. $I_2$ is saturated w.r.t. all
variables because all variables are invertible modulo $I_2$. Now $I_A$
can be computed from $I_2$ by eliminating the variables
$t,y_1,\ldots,y_m$.
\begin{rawhtml}
<!-- l2h_end sing_l2h 339 -->

\end{rawhtml}
\begin{rawhtml}

<!-- l2h_begin sing_l2h 340 -->
\end{rawhtml}
basis $v_1,\ldots,v_r$ for the integer kernel of $A$ using the
LLL-algorithm. The ideal corresponding to the lattice basis vectors
$$ I_1=<x^{v_i^+}-x^{v_i^-}|i=1,\ldots,r> $$
is saturated -- as in the algorithm of Conti and Traverso -- by
inversion of all variables: One adds an auxiliary variable $t$ and the
generator $t\cdot x_1\cdot\ldots\cdot x_n -1$ to obtain an ideal $I_2$
in $K[t,x_1,\ldots,x_n]$ from which one computes $I_A$ by elimination of
$t$.
\begin{rawhtml}
<!-- l2h_end sing_l2h 340 -->

\end{rawhtml}
\begin{rawhtml}

<!-- l2h_begin sing_l2h 341 -->
\end{rawhtml}
compute $I_A$ without any auxiliary variables, provided that $A$ contains a vector $w$
with positive coefficients in its row space. This is a real restriction,
i.e., the algorithm will not necessarily work in the general case.

A lattice basis $v_1,\ldots,v_r$ is again computed via the
LLL-algorithm. The saturation step is performed in the following way:
First note that $w$ induces a positive grading w.r.t. which the ideal
$$ I=<x^{v_i^+}-x^{v_i^-}|i=1,\ldots,r> $$
corresponding to our lattice basis is homogeneous. We use the following
lemma:

Let $I$ be a homogeneous ideal w.r.t. the weighted reverse
lexicographical ordering with weight vector $w$ and variable order $x_1
> x_2 > \ldots > x_n$. Let $G$ denote a Groebner basis of $I$ w.r.t. to
this ordering.  Then a Groebner basis of $(I:x_n^\infty)$ is obtained by
dividing each element of $G$ by the highest possible power of $x_n$.

From this fact, we can successively compute
$$ I_A= I:(x_1\cdot\ldots\cdot x_n)^\infty
=(((I:x_1^\infty):x_2^\infty):\ldots :x_n^\infty); $$
in the $i$-th step we take $x_i$ as the cheapest variable and apply the
lemma with $x_i$ instead of $x_n$.

This procedure involves $n$ Groebner basis computations. Actually, this
number can be reduced to at most $n/2$
\begin{rawhtml}
<!-- l2h_end sing_l2h 341 -->

\end{rawhtml}
\begin{rawhtml}

<!-- l2h_begin sing_l2h 342 -->
\end{rawhtml}
to $n/2$ Groebner basis
computations. It needs no auxiliary variables, but a supplementary
precondition; namely, the existence of a vector without zero components
in the kernel of $A$.

The main idea comes from the following observation:

Let $B$ be an integer matrix, $u_1,\ldots,u_r$ a lattice basis of the
integer kernel of $B$. Assume that all components of $u_1$ are
positive. Then
$$ I_B=<x^{u_i^+}-x^{u_i^-}|i=1,\ldots,r>, $$
i.e., the ideal on the right is already saturated w.r.t. all variables.

The algorithm starts by finding a lattice basis $v_1,\ldots,v_r$ of the
kernel of $A$ such that $v_1$ has no zero component. Let
$\{i_1,\ldots,i_l\}$ be the set of indices $i$ with
$v_{1,i}<0$. Multiplying the components $i_1,\ldots,i_l$ of
$v_1,\ldots,v_r$ and the columns $i_1,\ldots,i_l$ of $A$ by $-1$ yields
a matrix $B$ and a lattice basis $u_1,\ldots,u_r$ of the kernel of $B$
that fulfill the assumption of the observation above. We are then able
to compute a generating set of $I_A$ by applying the following
``variable flip'' successively to $i=i_1,\ldots,i_l$:

Let $>$ be an elimination ordering for $x_i$. Let $A_i$ be the matrix
obtained by multiplying the $i$-th column of $A$ with $-1$. Let
$$\{x_i^{r_j} x^{a_j} - x^{b_j} | j\in J \}$$
be a Groebner basis of $I_{A_i}$ w.r.t. $>$ (where $x_i$ is neither
involved in $x^{a_j}$ nor in $x^{b_j}$). Then
$$\{x^{a_j} - x_i^{r_j} x^{b_j} | j\in J \}$$
is a generating set for $I_A$.
\begin{rawhtml}
<!-- l2h_end sing_l2h 342 -->

\end{rawhtml}
\begin{rawhtml}

<!-- l2h_begin sing_l2h 343 -->
\end{rawhtml}
variable $u$ and one supplementary generator $x_1\cdot\ldots\cdot x_n -
u$ (instead of the generator $t\cdot x_1\cdot\ldots\cdot x_n -1$ in
the algorithm of Pottier). The algorithm uses a quite unusual technique to
get rid of the variable $u$ again.
\begin{rawhtml}
<!-- l2h_end sing_l2h 343 -->

\end{rawhtml}
\begin{rawhtml}

<!-- l2h_begin sing_l2h 344 -->
\end{rawhtml}
Let $A$ be an $m\times n$ matrix with integral coefficients, $b\in
Z\!\!\! Z^m$ and $c\in Z\!\!\! Z^n$. The problem
$$ \min\{c^T x | x\in Z\!\!\! Z^n, Ax=b, x\geq 0\hbox{
component-wise}\} $$
is called an instance of the \bf integer programming problem \rm or
\bf IP problem. \rm

The IP problem is very hard; namely, it is NP-complete.

For the following discussion let $c\geq 0$ (component-wise). We
consider $c$ as a weight vector; because of its non-negativity, $c$ can
be refined into a monomial ordering $>_c$. It turns out that we can
solve such an IP instance with the help of toric ideals:

First we assume that an initial solution $v$ (i.e., $v\in Z\!\!\!
Z^n, v\geq 0, Av=b$) is already known. We obtain the optimal solution
$v_0$ (i.e., with $c^T v_0$ minimal) by the following procedure:
\begin{rawhtml}
<!-- l2h_end sing_l2h 344 -->

\end{rawhtml}
\begin{rawhtml}

<!-- l2h_begin sing_l2h 345 -->
\end{rawhtml}
$>_c$
\begin{rawhtml}
<!-- l2h_end sing_l2h 345 -->

\end{rawhtml}
\begin{rawhtml}

<!-- l2h_begin sing_l2h 346 -->
\end{rawhtml}
$x^v$
\begin{rawhtml}
<!-- l2h_end sing_l2h 346 -->

\end{rawhtml}
\begin{rawhtml}

<!-- l2h_begin sing_l2h 347 -->
\end{rawhtml}
$x^(v_0)$
\begin{rawhtml}
<!-- l2h_end sing_l2h 347 -->

\end{rawhtml}
\begin{rawhtml}

<!-- l2h_begin sing_l2h 348 -->
\end{rawhtml}
$v_0$
\begin{rawhtml}
<!-- l2h_end sing_l2h 348 -->

\end{rawhtml}
\begin{rawhtml}

<!-- l2h_begin sing_l2h 349 -->
\end{rawhtml}
Faug\`ere,
\begin{rawhtml}
<!-- l2h_end sing_l2h 349 -->

\end{rawhtml}
\begin{rawhtml}

<!-- l2h_begin sing_l2h 350 -->
\end{rawhtml}
$ 2*genus-2 < deg(G) < size(D) $
\begin{rawhtml}
<!-- l2h_end sing_l2h 350 -->

\end{rawhtml}
\begin{rawhtml}

<!-- l2h_begin sing_l2h 351 -->
\end{rawhtml}
$\Omega(G-D)$
\begin{rawhtml}
<!-- l2h_end sing_l2h 351 -->

\end{rawhtml}
\begin{rawhtml}

<!-- l2h_begin sing_l2h 352 -->
\end{rawhtml}
$epsilon$
\begin{rawhtml}
<!-- l2h_end sing_l2h 352 -->

\end{rawhtml}
\begin{rawhtml}

<!-- l2h_begin sing_l2h 353 -->
\end{rawhtml}
$delta$
\begin{rawhtml}
<!-- l2h_end sing_l2h 353 -->

\end{rawhtml}
\begin{rawhtml}

<!-- l2h_begin sing_l2h 354 -->
\end{rawhtml}
$epsilon + genus$
\begin{rawhtml}
<!-- l2h_end sing_l2h 354 -->

\end{rawhtml}
\begin{rawhtml}

<!-- l2h_begin sing_l2h 355 -->
\end{rawhtml}
$epsilon:=[(deg(G)-3*genus+1)/2]$
\begin{rawhtml}
<!-- l2h_end sing_l2h 355 -->

\end{rawhtml}
\end{document}
